\documentclass[12pt]{beamer}
\usepackage{amsmath}
\usepackage{amssymb}

\begin{document}

%----------------------------------------------------------------------%
\section{The Geometric Distribution}
\begin{frame}
\huge
\[ \mbox{Statistics and Probability} \]
\huge
\[ \mbox{The Geometric Distribution} \]
\Large
\[ \mbox{www.Stats-Lab.com} \]

\end{frame}
%----------------------------------------------------------------------------------%
\begin{frame}
\frametitle{The Geometric Distribution}
\Large
\begin{itemize}
\item The geometric distribution is used for Bernouilli Trials, where there outcome are classified as either failures or successes.
\end{itemize} \bigskip

In probability theory, the geometric distribution is either of two discrete probability distributions:
\begin{itemize}
\item The probability distribution of the number of trials needed to get first success, supported on the set $\{ 1, 2, 3, \ldots\}$
\item The probability distribution of the number of failures before the first success, supported on the set $\{ 0, 1, 2, 3, \ldots\}$
\end{itemize}
\end{frame}

%----------------------------------------------------------------------------------%
\begin{frame}
\frametitle{The Geometric Distribution}
\Large
\begin{itemize}
\item Which of these one calls "the" geometric distribution is a matter of convention and convenience. A solution for one can quickly be surmised from the other.
\item These two different geometric distributions should not be confused with each other. 
\item Often, the name shifted geometric distribution is adopted for the former one (distribution of the number X); 
\item however, to avoid ambiguity, it is considered wise to indicate which is intended, by mentioning the support explicitly.
\end{itemize}
\end{frame}
%----------------------------------------------------------------------------------%
\begin{frame}
\frametitle{The Geometric Distribution}
\Large
\begin{itemize}
\item It’s the probability that the first occurrence of success require k number of independent trials, each with success probability p. 

\item If the probability of success on each trial is p, then the probability that the kth trial (out of k trials) is the first success is
\[  P(X = k) = (1-p)^{k-1}\,p\, \phantom{space} for k = 1, 2, 3, \ldots \]


\item The above form of geometric distribution is used for modeling the number of trials until the first success. 
\end{itemize}
\end{frame}
%----------------------------------------------------------------------------------%
\begin{frame}
\frametitle{The Geometric Distribution}
\Large
\begin{itemize}
\item By contrast, the following form of geometric distribution is used for modeling number of failures until the first success:
\[ P(Y=k) = (1 - p)^k\,p\, \phantom{space} for k = 0, 1, 2, 3, \ldots\]
\end{itemize}
\end{frame}
%----------------------------------------------------------------------------------%
\begin{frame}
\frametitle{The Geometric Distribution}
\Large
\begin{itemize}
\item
In either case, the sequence of probabilities is a geometric sequence.

\itemm For example, suppose an ordinary die is thrown repeatedly until the first time a "1" appears. 
\item The probability distribution of the number of times it is thrown is supported on the infinite set ${ 1, 2, 3, \ldots }$ and is a geometric distribution with p = 1/6.
\end{itemize}
\end{frame}
%----------------------------------------------------------------------------------%
\begin{frame}
\frametitle{The Geometric Distribution}
\Large
\begin{itemize}
\item

The expected value of a geometrically distributed random variable X is 1/p and the variance is $(1 - p)/p^2$:
\[ \mathrm{E}(X) = \frac{1}{p}, \qquad\mathrm{var}(X) = \frac{1-p}{p^2}. \]
\item Similarly, the expected value of the geometrically distributed random variable Y (where Y corresponds to the pmf listed in the right column) is (1 - p)/p, 
and its variance is (1 - p)/p2:
\[ \mathrm{E}(Y) = \frac{1-p}{p}, \qquad\mathrm{var}(Y) = \frac{1-p}{p^2}.\]

\end{itemize}
\end{frame}
%-------------------------------------------------------------------------------%
\section{geometric Distribution - Example}
\begin{frame}
\frametitle{The geometric ditribution}

Suppose that I am at a party and I start asking girls to dance. Let X be the number of girls that I need to ask in order to find a partner. If the first girl accepts, then X=1. If the first girl declines but the next girl accepts, then X=2. And so on. 

When X=n, it means that I failed on the first n-1 tries and succeeded on the nth try. My probability of failing on the first try is (1-p). My probabilty of failing on the first two tries is (1-p)(1-p). 
\end{frame}
%-------------------------------------------------------------------------------%
\begin{frame}
\frametitle{The Geometric ditribution}
My probability of failing on the first n-1 tries is (1-p)n-1. Then, my probability of succeeding on the nth try is p. Thus, we have 

\[ P(X = n) = (1-p)n-1p \]

This is known as the geometric distribution. When you have a sequence of numbers in which the (n+1)th number is a multiple of the nth number, it is called a geometric sequence. In this case, P(X = n+1) is a multiple of P(X = n). (What is that multiple?) 
\end{frame}
%-------------------------------------------------------------------------------%
\begin{frame}
\frametitle{The geometric ditribution}
What is the probability that it will take more than n tries to succeed? We know that if I ask an infinite number of girls to dance, eventually one of them will accept. So, the probability that it will take more than n tries is the same as the probability that I fail n times. That is, 

\[ P(X > n) = (1-p)n \]
\end{frame}
%-------------------------------------------------------------------------------%
\begin{frame}
\frametitle{The geometric ditribution}
If X is geometric with parameter p, what is E(X)? 

We are faced with an infinite sum. Multiplying X times P(X) for X = 1, 2, 3, ... gives 

\[
[1] S = p + 2p(1-p) + 3p(1-p)2 +...+np(1-p)n-1 
\]
Multiply both sides by (1-p) and you have

[2] (1-p)S = p(1-p) + 2p(1-p)2 + 3p(1-p)3 +...+np(1-p)n 
\end{frame}
%-------------------------------------------------------------------------------%
\begin{frame}
\frametitle{The geometric ditribution}
Subtracting [2] from [1] gives 

S - (1-p)S = pS = p[1 + (1-p) + (1-p)2 + ...(1-p)n] = p(1/p) = 1
S = 1/p 

Therefore, the mean of the geometric distribution is equal to 1/p. If we are trying to estimate how many girls I will have to ask to dance until I find a partner, and p, the probability of one girl accepting, is .2, then on average I will have to ask five girls. 
You will not have to know it, but for the record, the variance of the geometric distribution is (1-p)/p2. 
\end{frame}
%-------------------------------------------------------------------------------%
\end{document}