
\documentclass[12pt, a4paper]{report}
\usepackage{natbib}
\usepackage{vmargin}
\usepackage{graphicx}
\usepackage{epsfig}
\usepackage{subfigure}
%\usepackage{amscd}
\usepackage{amssymb}
\usepackage{subfigure}
\usepackage{amsbsy}
\usepackage{amsthm, amsmath}
%\usepackage[dvips]{graphicx}
\bibliographystyle{chicago}
\renewcommand{\baselinestretch}{1.8}

% left top textwidth textheight headheight % headsep footheight footskip
\setmargins{3.0cm}{2.5cm}{15.5 cm}{23.5cm}{0.5cm}{0cm}{1cm}{1cm}

\pagenumbering{arabic}


\begin{document}
\author{Kevin O'Brien}
\title{Transfer Report}
\date{\today}
\maketitle

\tableofcontents \setcounter{tocdepth}{2}

\newpage
\chapter{Introduction}

\section{Introduction}

\chapter{Proposal of Research}



\section{LME models in method comparison studies}
%With the greater computing power available for scientific
%analysis, it is inevitable that complex models such as linear
%mixed effects models should be applied to method comparison
%studies.

Linear mixed effects (LME) models can facilitate greater
understanding of the potential causes of bias and differences in
precision between two sets of measurement. \citet{LaiShiao} views
the uses of linear mixed effects models as an expansion on the
Bland-Altman methodology, rather than as a replacement.
\citet{BXC2008} remarks that modern statistical computation, such
as that used for LME models, greatly improve the efficiency of
calculation compared to previous `by-hand' methods. In this
chapter various LME approaches to method comparison studies shall
be examined.

\newpage

\section{Roy's LME methodology for assessing agreement}

\citet{ARoy2009} proposes the use of LME models to perform a test
on two methods of agreement to determine whether they can be used
interchangeably.

Bivariate correlation coefficients have been shown to be of
limited use in method comparison studies \citep{BA86}. However,
recently correlation analysis has been developed to cope with
repeated measurements, enhancing their potential usefulness. Roy
incorporates the use of correlation into his methodology.

Roy's method considers two methods to be in agreement if three
conditions are met.

\begin{itemize}
\item no significant bias, i.e. the difference between the two
mean readings is not "statistically significant",

\item high overall correlation coefficient,

\item the agreement between the two methods by testing their
repeatability coefficients.

\end{itemize}

The methodology uses a linear mixed effects regression fit using
compound symmetry (CS) correlation structure on \textbf{V}.


$\Lambda = \frac{\mbox{max}_{H_{0}}L}{\mbox{max}_{H_{1}}L}$

\newpage

\citet{ARoy2009} considers the problem of assessing the agreement
between two methods with replicate observations in a doubly
multivariate set-up using linear mixed effects models.

\citet{ARoy2009} uses examples from \citet{BA86} to be able to
compare both types of analysis.

\citet{ARoy2009} proposes a LME based approach with Kronecker
product covariance structure with doubly multivariate setup to
assess the agreement between two methods. This method is designed
such that the data may be unbalanced and with unequal numbers of
replications for each subject.

The maximum likelihood estimate of the between-subject variance
covariance matrix of two methods is given as $D$. The estimate for
the within-subject variance covariance matrix is $\hat{\Sigma}$.
The estimated overall variance covariance matrix `Block
$\Omega_{i}$' is the addition of $\hat{D}$ and $\hat{\Sigma}$.


\begin{equation}
\mbox{Block  }\Omega_{i} = \hat{D} + \hat{\Sigma}
\end{equation}

For the the RV-IC comparison, $\hat{D}$ is given by


\begin{equation}
\hat{D}= \left[ \begin{array}{cc}
  1.6323 & 1.1427  \\
  1.1427 & 1.4498 \\
\end{array} \right]
\end{equation}

The estimate for the within-subject variance covariance matrix is
given by
\begin{equation}
\hat{\Sigma}= \left[ \begin{array}{cc}
  0.1072 & 0.0372  \\
  0.0372 & 0.1379  \\
\end{array}\right]
\end{equation}
The estimated overall variance covariance matrix for the the 'RV
vs IC' comparison is given by
\begin{equation}
Block \Omega_{i}= \left[ \begin{array}{cc}
  1.7396 & 1.1799  \\
  1.1799 & 1.5877  \\
\end{array} \right].
\end{equation}

 The power of the
likelihood ratio test may depends on specific sample size and the
specific number of  replications, and the author proposes
simulation studies to examine this further.


% \begin{equation}
% data here
% \end{equation}

\newpage
\section{Lai Shiao}
\citet{LaiShiao} use mixed models to determine the factors that
affect the difference of two methods of measurement using the
conventional formulation of linear mixed effects models.

If the parameter \textbf{b}, and the variance components are not
significantly different from zero, the conclusion that there is no
inter-method bias can be drawn. If the fixed effects component
contains only the intercept, and a simple correlation coefficient
is used, then the estimate of the intercept in the model is the
inter-method bias. Conversely the estimates for the fixed effects
factors can advise the respective influences each factor has on
the differences. It is possible to pre-specify different
correlation structures of the variance components \textbf{G} and
\textbf{R}.


Oxygen saturation is one of the most frequently measured variables
in clinical nursing studies. `Fractional saturation' ($HbO_{2}$)
is considered to be the gold standard method of measurement, with
`functional saturation' ($SO_{2}$) being an alternative method.
The method of examining the causes of differences between these
two methods is applied to a clinical study conducted by
\citet{Shiao}. This experiment was conducted by 8 lab
practitioners on blood samples, with varying levels of
haemoglobin, from two donors. The samples have been in storage for
varying periods ( described by the variable `Bloodage') and are
categorized according to haemoglobin percentages(i.e
$0\%$,$20\%$,$40\%$,$60\%$,$80\%$,$100\%$). There are 625
observations in all.

\citet{LaiShiao} fits two models on this data, with the lab
technicians and the replicate measurements as the random effects
in both models. The first model uses haemoglobin level as a fixed
effects component. For the second model, blood age is added as a
second fixed factor.

\subsubsection{Single fixed effect} The first model fitted by \citet{LaiShiao} takes the
blood level as the sole fixed effect to be analyzed. The following
coefficient estimates are estimated by `Proc Mixed';
\begin{eqnarray}
\mbox{fixed effects :   } 2.5056 - 0.0263\mbox{Fhbperct}_{ijtl} \\
(\mbox{p-values :   } = 0.0054, <0.0001, <0.0001)\nonumber\\\nonumber\\
\mbox{random effects :   } u(\sigma^{2}=3.1826) + e_{ijtl}
(\sigma^{2}_{e}=0.1525, \rho= 0.6978) \nonumber\\
(\mbox{p-values :   } = 0.8113, <0.0001, <0.0001)\nonumber
\end{eqnarray}

With the intercept estimate being both non-zero and statistically
significant ($p=0.0054$), this models supports the presence
inter-method bias is $2.5\%$ in favour of $SO_{2}$. Also, the
negative value of the haemoglobin level coefficient indicate that
differences will decrease by $0.0263\%$ for every percentage
increase in the haemoglobin .

In the random effects estimates, the variance due to the
practitioners is $3.1826$, indicating that there is a significant
variation due to technicians ($p=0.0311$) affecting the
differences. The variance for the estimates is given as $0.1525$,
($p<0.0001$).

\subsubsection{Two fixed effects}
Blood age is added as a second fixed factor to the model,
whereupon new estimates are calculated;
\begin{eqnarray}
\mbox{fixed effects :   } -0.2866 + 0.1072 \mbox{Bloodage}_{ijtl}
- 0.0264\mbox{Fhbperct}_{ijtl}\nonumber\\
( \mbox{p-values :   } = 0.8113, <0.0001, <0.0001)\nonumber\\\nonumber\\
\mbox{random effects :   } u(\sigma^{2}=10.2346) + e_{ijtl}
(\sigma^{2}_{e}=0.0920, \rho= 0.5577) \nonumber\\
(\mbox{p-values :   } = 0.0446, <0.0001, <0.0001)
\end{eqnarray}


With this extra fixed effect added to the model, the intercept
term is no longer statistically significant. Therefore, with the
presence of the second fixed factor, the model is no longer
supporting the presence of inter-method bias. Furthermore, the
second coefficient indicates that the blood age of the observation
has a significant bearing on the size of the difference between
both methods ($p <0.0001$). Longer storage times for blood will
lead to higher levels of particular blood factors such as MetHb
and HbCO (due to the breakdown and oxidisation of the
haemoglobin). Increased levels of MetHb and HbCO are concluded to
be the cause of the differences. The coefficient for the
haemoglobin level doesn't differ greatly from the single fixed
factor model, and has a much smaller effect on the differences.
The random effects estimates also indicate significant variation
for the various technicians; $10.2346$ with $p=0.0446$.

\citet{LaiShiao} demonstrates how that linear mixed effects models
can be used to provide greater insight into the cause of the
differences. Naturally the addition of further factors to the
model provides for more insight into the behavior of the data.






