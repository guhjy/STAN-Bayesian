%% MA4102 Lectures (Week 12)
\end{frame}
%=========================================%
\begin{frame}
Remarks-  For continuous distributions only.

The probability that a continuous random variable will take an exact value is infinitely small.

We will usually treat it as if it was zero. 

When we write probabilities in mathematical notation, we often retain the equality component (i.e. the "...or equal to..").
For example, we would write expressions  $P(X \geq 2)$ or $P(X \leq 5)$ . 

Because the probability of an exact value is almost zero, these two expression are equivalent to   $P(X > 2)$ or $P(X<5)$  .

The complement of  $P(X \geq k)$  can be written as $P(X \leq k)$   . 
\end{frame}

\frametitle{The Standard Normal (“z”) Distribution}
 
A random variable that has a normal distribution with a mean of zero and a standard deviation of one is said to have a standard normal probability distribution.  It is often nick-named the "z" distribution.
 
Importantly, probabilities relating to the z distribution are comprehensively tabulated in Murdoch Barnes table 3.

Given a value of k (with k between 0 and 4), the probability of a  standard normal "z" random variable being greater than (or equal to) k is given in Murdoch Barnes table 3 (page 71). 

$P(X \leq k)$

\end{frame}
%==========================================================%
\begin{frame}
Example 1  

Find  $P(Z geq 1.8)$

note  1.8 = 1.8 + 0.00

    The row is 1.8
    The column is 0.00


%% Graphic %% 
\end{frame}
%==========================================================%
\begin{frame}


Answer   = 0.0359




\end{frame}
%==========================================================%
\begin{frame}
Example 2  

Find  (P X \geq 1.96)$

note  1.96 = 1.9 + 0.06

    The row is 1.9
    The column is 0.06
\end{frame}
%==========================================================%
\begin{frame}


\frametitle{Complement and Symmetry Rules}

Any normal distribution problem can be solved with some combination of the following rules.
\end{frame}
%==============================================%
\begin{frame} 
a. The Complement rule

(Common to all continuous random variables) 

 

Example

   


\end{frame}
%==============================================%
\begin{frame}
b. The Symmetry rule
 
This rule is based on the property of symmetry mentioned previously.

 

Only the probabilities corresponding to values between 0 and 4 are tabulated in Murdoch Barnes.

If we have a negative value of k, we can use the symmetry rule.  
Example 

Find  
Solution 

 

Example 
Find the probability of a "z" random variable greater than (or equal to) -1.8?
Find  

Solution

 
\end{frame}
%==============================================%
\begin{frame}
Example 

Find the probability of a "z" random variable being between -1.8 and 1.96?
Find  

Solution 

Consider the complment event of being in this interval: a combination of being too low or two high. 

The probability of being too low for this interval is    (from before)

The probability of being too high for this interval is    (from before)

Therefore the probability of being outside the interval is  0.0359 + 0.0250 = 0.0609.

Therefore the probability of being inside the interval is 1- 0.0609 = 0.9391

  = 0.9391


\end{frame}
%==============================================%
\begin{frame}
Solving using the z distribution
When we have a normal distribution with any mean   and any standard deviation  , we answer probability questions about the distribution by first converting all values to corresponding values of the standard normal ("z") distribution. 

The formula used to convert any random variable "x" ( with mean   and standard deviation   specified)  to the standard normal ("z") distribution is given as follows. 

 
  is the standard normal random variable with a mean of zero and a standard deviation of 1. 

It can be thought of as a measure of how many standard deviations that a value "x" is from mean  .

Remarks

A value of x equal to mean   results in a z -value of 0
 

Thus we can see that a value of "x" corresponding to its mean    corresponds to a z-value at its mean , which is 0.

A value of "x" that is one standard deviation above its mean (i.e.  ), we see that the corresponding z value is 1.


Thus a value of x that is one standard deviation away from it's mean yields a z-value of 1.

Example
Given that the mean   = 100 and that the standard deviation    = 2.5, what is the "z-value" for normal random variable x = 106?

 

Relationship between "x value" and "z  value"

[VERY IMPORTANT]

If    		(   and   are some values )

then  


By extension  


From previous example 

 

From Murdoch Barnes Table 3, 

 


Therefore 
 




________________________________________
[Finish] 

 
