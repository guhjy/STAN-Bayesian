\documentclass[IntroMain.tex]{subfiles} 
\begin{document}
	
%------------------------------------------------------------------%
\frame{
	\frametitle{Poisson Expected Value and Variance}
	
	
	If the random variable X has a Poisson distribution with parameter $m$, we write
	\[ X \sim Poisson(m) \]
	
	
	\begin{itemize}
		\item Expected Value of X : E(X) = m
		\item Variance of X : $\mbox{Var}(X) = m$
		\item Standard Deviation of X : $SD(X) = \sqrt{m}$
	\end{itemize}
}
%------------------------------------------------------------------%
\frame{
	\frametitle{Poisson Distribution : Example} 
	
	\begin{itemize}
		\item The number of faults in a fibre optic cable were recorded for each kilometre length of cable.
		\item The mean number of faults was found to be 4 faults per kilometre.
		\item The standard deviation of the number of faults was found to be 2 faults per kilometre.
		\item Is the Poisson Distribution is a useful technique for modelling the number of faults in fibre optic cable?
		\item (Looking at the last slide, the answer is yes, because the variance and mean are equal). 
	\end{itemize}
	
}
%---------------------------------------------------------------------%
\begin{frame}
	\frametitle{Poisson Approximation of the Binomial}
	\begin{itemize}
		\item The Poisson distribution can sometimes be used to approximate the
		binomial distribution
		\item When the number of observations n is large, and the success probability
		p is small, the $B(n,p)$ distribution approaches the Poisson distribution
		with the parameter given by $m = np$.
		\item This is useful since the computations involved in calculating binomial
		probabilities are greatly reduced.
		\item As a rule of thumb, n should be greater than 50 with p very small, such
		that np should be less than 5.
		\item If the value of p is very high, the definition of what constitutes a
		``success" or ``failure" can be switched.
	\end{itemize}
\end{frame}

%---------------------------------------------------------------------%
\begin{frame}
	\frametitle{Poisson Approximation: Example}
	
	\begin{itemize}
		\item Suppose we sample 1000 items from a production line that is producing, on
		average, $0.1\%$ defective components.
		\item Using the binomial distribution, the probability of exactly 3 defective items in
		our sample is
		\[P(X = 3) = ^{1000}C_{3} \times 0.001^{3} \times 0.999^{997}\]
	\end{itemize}
\end{frame}

%---------------------------------------------------------------------%
\begin{frame}
	\frametitle{Poisson Approximation: Example}
	Lets compute each of the component terms individually.
	
	\begin{itemize}
		\item $^{1000}C_{3}$
		\[^{1000}C_{3} = \frac{1000 \times 999 \times 998}{3 \times 2 \times 1} = 166,167,000\]
		\item $0.001^3$
		\[0.001^3 = 0.000000001\]
		\item $0.999^{997}$
		\[0.999^{997} = 0.36880\]
	\end{itemize}
	
	
	Multiply these three values to compute the binomial probability
	$P(X = 3) = 0.06128$
\end{frame}

\begin{frame}
	\frametitle{Poisson Approximation: Example}
	\begin{itemize}
		\item Lets use the Poisson distribution to approximate a solution.
		\item First check that $n \geq 50$ and $np < 5$ (Yes to both).
		\item We choose as our parameter value $m = np = 1000 \times 0.001 = 1$
	\end{itemize}
	\[P(X = 3) = \frac{e^{-1} \times 1^3}{3!} = \frac{e^{-1}}{6} = \frac{0.36787}{6} = 0.06131 \]
	Compare this answer with the Binomial probability
	P(X = 3) = 0.06128.
	Very good approximation, with much less computation effort.
\end{frame}
%---------------------------------------------------------------------%
\begin{frame}
	\frametitle{Poisson Approximation of the Binomial}
	
	\begin{itemize}
		\item The Poisson distribution can sometimes be used to approximate the
		binomial distribution
		\item When the number of observations n is large, and the success probability
		p is small, the $Bin(n,p)$ distribution approaches the Poisson distribution
		with the parameter given by $m = np$.
		\item This is useful since the computations involved in calculating binomial
		probabilities are greatly reduced.
		\item As a rule of thumb, n should be greater than 50 with p very small, such
		that np should be less than 5.
		\item If the value of p is very high, the definition of what constitutes a
		``success" or ``failure" can be switched.
	\end{itemize}
\end{frame}

%---------------------------------------------------------------------%
\begin{frame}
	\frametitle{Poisson Approximation: Example}
	
	\begin{itemize}
		\item Suppose we sample 1000 items from a production line that is producing, on
		average, $0.1\%$ defective components.
		\item Using the binomial distribution, the probability of exactly 3 defective items in
		our sample is
		\[P(X = 3) = ^{1000}C_{3} \times 0.001^{3} \times 0.999^{997}\]
	\end{itemize}
\end{frame}

%---------------------------------------------------------------------%
\begin{frame}
	\frametitle{Poisson Approximation: Example}
	Lets compute each of the component terms individually.
	
	\begin{itemize}
		\item $^{1000}C_{3}$
		\[^{1000}C_{3} = \frac{1000 \times 999 \times 998}{3 \times 2 \times 1} = 166,167,000\]
		\item $0.001^3$
		\[0.001^3 = 0.000000001\]
		\item $0.999^{997}$
		\[0.999^{997} = 0.36880\]
	\end{itemize}
	
	
	Multiply these three values to compute the binomial probability
	$P(X = 3) = 0.06128$
\end{frame}

\begin{frame}
	\frametitle{Poisson Approximation: Example}
	\begin{itemize}
		\item Lets use the Poisson distribution to approximate a solution.
		\item First check that $n \geq 50$ and $np < 5$ (Yes to both).
		\item We choose as our parameter value $m = np = 1000 \times 0.001 = 1$
	\end{itemize}
	\[P(X = 3) = \frac{e^{-1} \times 1^3}{3!} = \frac{e^{-1}}{6} = \frac{0.36787}{6} = 0.06131 \]
	Compare this answer with the Binomial probability
	P(X = 3) = 0.06128.
	Very good approximation, with much less computation effort.
\end{frame}
%---------------------------------------------------------------------%
\end{document}
