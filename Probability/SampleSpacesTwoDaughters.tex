\documentclass[IntroMain.tex]{subfiles} 
\begin{document}


\begin{frame}
\frametitle{Sample Spaces}
\Large
Consider couplse that have two children. Treating the gender of the children as an \textit{\textbf{ordered pair}} outcome of a random experiment, the sample space is 
\[\boldsymbol{S} = \{ (b,b), (b,g), (g,b), (g,g)\}.\]
Let us assume that each sample point is \textit{\textbf{equiprobable}}, with probability 0.25 for each sample point.


\end{frame}

\begin{frame}
\frametitle{Sample Spaces}
\Large
\vspace{-1cm}
 Find the probability that both children are girls if it is known that: 

\begin{itemize}
\item[(a)] at least one of the children is a girl,
\item[(b)] the older child is a girl. 	
\end{itemize}	
\end{frame}


\begin{frame}
\frametitle{Sample Spaces}
\Large
\vspace{-1.5cm}
\textbf{Part a} \\
 Find the probability that both children are girls if it is known that at least one of the children is a girl.
\[\boldsymbol{S} = \{ (b,b), (b,g), (g,b), (g,g)\}.\]
	
\end{frame}

\begin{frame}
\frametitle{Sample Spaces}
\Large
\vspace{-1.5cm}
\textbf{Part b} \\
 Find the probability that both children are girls if it is known that the older child is a girl.
\[\boldsymbol{S} = \{ (b,b), (b,g), (g,b), (g,g)\}.\]
	
\end{frame}

\end{document}
