\documentclass[IntroMain.tex]{subfiles} 
\begin{document}

%--------------------------------------------------------------------------------%
\begin{frame}
\frametitle{Probability}
\Large
If there are $n$ possible outcomes to an experiment, and m ways in which event
A can happen, then the probability of event A ( which we write as P(A)) is
\[ P(A) = \frac{m}{n} \]

\end{frame}
%======================================================================= %
\begin{frame}
	\frametitle{Probability}
The probability of the event A may be interpreted as the proportion of times
that event A will occur if we repeat the random experiment an infinite number
of times.\\ \bigskip

\end{frame}
%======================================================================= %
\begin{frame}
\textbf{Rules}:\\
\begin{itemize}
\item[1] $0 \leq P(A) \leq 1 $: the probability of any event lies between 0 and 1
inclusive.
\item[2] $P(S) = 1$ : the probability of the sample space is always equal to 1.
\item[3] $P(A^c) = 1-P(A)$ : how to compute the probability of the complement.
\end{itemize}
\end{frame}


%--------------------------------------------------------%

\end{document}