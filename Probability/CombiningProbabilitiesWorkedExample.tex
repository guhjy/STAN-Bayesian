\documentclass[IntroMain.tex]{subfiles} 
\begin{document}
	%--------------------------------------------------------------------------------------%
%--------------------------------------------------------------------------------------%
\begin{frame}
	\textbf{Combining Probabilities - Worked Example}
A new test has been developed to diagnose a particular disease. If a person has the disease, the test has a 95\% chance of identifying them as having the disease. 

If a person does not have the disease, the test has a 1\% chance of identifying them as having the disease. 5\% of the population have this disease. Suppose we select a person at random from the population.
\end{frame}
%--------------------------------------------------------------------------------------%
%--------------------------------------------------------------------------------------%
\begin{frame}
	\textbf{Combining Probabilities - Worked Example}
Q1 - What is the probability that the test will identify them as having the disease?

Q2 - What is the probability that the person has the disease given that the test identifies them as having the disease?

\end{frame}
%--------------------------------------------------------------------------------------%
%--------------------------------------------------------------------------------------%
\begin{frame}
	\textbf{Combining Probabilities - Worked Example}
						
Solutions 
\begin{itemize}
\item	Let P signify that a test will give a “positive” result 
\item	Let N signify that a test will give a “negative” result	
\item	Let D signify that the person in question has the disease
\item	Let H signify that the person doesn’t have the disease ( or in other words , is healthy) 
\end{itemize}
\end{frame}
%--------------------------------------------------------------------------------------%
%--------------------------------------------------------------------------------------%
\begin{frame}
	\textbf{Combining Probabilities - Worked Example}
We are asked to determine the following 
1) The probability of a positive test - p(P)
2) The probability that they have the disease given that they have tested positive – p(D|P)
	
\end{frame}
%--------------------------------------------------------------------------------------%
%--------------------------------------------------------------------------------------%
\begin{frame}
	\textbf{Combining Probabilities - Worked Example} 
We are given the following three pieces of information
 

We know that D and H are complements, so we can work out the probabilities of these too. (P and N are complements also)

\end{frame}
%--------------------------------------------------------------------------------------%
%--------------------------------------------------------------------------------------%
\begin{frame}
	\textbf{Combining Probabilities - Worked Example}

People who test positive are made up of two groups
\begin{enumerate}
\item	People who test positive and who do have the disease  (P and D)
\item	People who test positive and who don’t have the disease  ( P and H)
\end{enumerate}
	\end{frame}
	%--------------------------------------------------------------------------------------%
	%--------------------------------------------------------------------------------------%
	\begin{frame}
		\textbf{Combining Probabilities - Worked Example}

Bayes Rule is given in the Formulae 		 

We can rearrange it as follows 		 

\end{frame}
%--------------------------------------------------------------------------------------%
%--------------------------------------------------------------------------------------%
\begin{frame}
	\textbf{Combining Probabilities - Worked Example}
We can now write our equation in terms of all the information we have :

   ANS

\end{frame}
%--------------------------------------------------------------------------------------%
%--------------------------------------------------------------------------------------%
\begin{frame}
	\textbf{Combining Probabilities - Worked Example}
For the second part, we simply use Bayes Rule again, using information we have determined previously

  ANS
  
\end{frame}
%--------------------------------------------------------------------------------------%
\end{document}
