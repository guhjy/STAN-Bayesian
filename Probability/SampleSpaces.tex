\documentclass[IntroMain.tex]{subfiles} 
\begin{document}
	%------------------------------------------------------------%
%=====================================================%
\begin{frame}
	\frametitle{Sample Spaces}
	
	% [http://cnx.org/content/m16845/latest/]
	\begin{itemize}
	\item	A complete list of all possible outcomes of a random experiment is called sample space or possibility space and is denoted by S.
		
		
	\item	A sample space is a set or collection of outcome of a particular random experiment.
		
	\item 	For example, imagine a dart board. You are trying to find the probability of getting a bullseye. The dart board is the sample space. The probability of a dart hitting the dart board is 1.0.
	\item For another example, imagine rolling a six sided die. The sample space is {1, 2, 3, 4, 5, 6}.
	\end{itemize}

	
\end{frame}
%=====================================================%
\begin{frame}
	\frametitle{Sample Spaces}
	\Large
	\begin{itemize}
\item	The following list consists of sample spaces of examples of random experiments and their respective outcomes.
	
\item	The tossing of a coin, sample space is {Heads, Tails}
	
\item 	The roll of a die, sample space is {1, 2, 3, 4, 5, 6}
	
\item The selection of a numbered ball (1-50) in an urn, sample space is $\{1, 2, 3, 4, 5, ...., 50\}$
	\end{itemize}
	
	
\end{frame}

%=====================================================%
\begin{frame}
	\frametitle{Sample Spaces}
	
	\begin{itemize}
	\item Percentage of calls dropped due to errors over a particular time period, sample space is $\{2\%, 14\%, 23\%, ......\}$
	
	\item The time difference between two messages arriving at a message centre, sample space is {0, ...., infinity}
	
	\item The time difference between two different voice calls over a particular network, sample space is {0, ...., infinity}
	\end{itemize}

		
	\end{frame}

	%=====================================================%
\end{document}
