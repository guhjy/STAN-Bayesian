\documentclass[IntroMain.tex]{subfiles} 
\begin{document}
	
\begin{frame}
\frametitle{Probability trees}
\begin{itemize}
\item The setting out of solutions to problems requiring the manipulation of the probabilities of mutually exclusive and independent events can sometimes be helped by the use of probability tree diagrams. These have useful applications in decision theory.

\item The best choice of probability tree structure often depends upon the question and the natural order in which events like A and B above occur.
\end{itemize}

\end{frame}
%--------------------------------------------------------- %
\begin{frame}
\frametitle{Probability Trees}
{
\large
\vspace{-0.3cm}
\begin{itemize}
\item Two gamblers, A and B, are playing each other in a tournament to win a jackpot worth $\$6,000$. 
\item The first gambler to win 5 rounds, wins the tournament, and the jackpot outright.
\item Each player has an equal chance of winning each round. Also, a tie is not possible.
\item The tournament is suspended after the seventh round. At this point A has won 3 rounds, while B has won 4.
\item They agree to finish then and divide up the jackpot, according to how likely an outright victory would have been for both.
\end{itemize}
How much money did A end up with?
}
\end{frame}

%---------------------------------------------- %
\begin{frame}
\frametitle{Probability Trees}
{  \large 
\vspace{-0.4cm}
\begin{itemize}
\item Consider that A needed to win two more rounds, while B only need to win one more.
\item One could suppose that B was twice as likely as A to win the jackpot.
\item That would mean that the shares of the jackpot would be $\$2,000$ for A and 
$\$4,000$ for B.
\end{itemize}
}
\end{frame}
\begin{frame}
\frametitle{Probability Trees}
{  \large 
\vspace{-0.4cm}
\begin{itemize}
\item Consider that A needed to win two more rounds, while B only need to win one more.
\item One could suppose that B was twice as likely as A to win the jackpot.
\item That would mean that the shares of the jackpot would be $\$2,000$ for A and 
$\$4,000$ for B.
\item \alert{WRONG!}
\end{itemize}
}
\end{frame}
\begin{frame}
\frametitle{Probability Trees}
{  \large 
\vspace{-0.4cm}
\begin{itemize}
\item At the end of the seventh round, A had a $25\%$ chance of winning the jackpot.
\item A's share of the jackpot is  the $\$1,500$.
\item B had a $75\%$ chance of winning, so gets $\$4,500$.
\end{itemize}
}
\end{frame}

\end{document}