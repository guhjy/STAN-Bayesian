\documentclass{beamer}

\usepackage{amsmath}


\begin{document}

\begin{frame}
\bigskip
{\Huge \[ \mbox{The Normal Distribution}\] }

{\huge \[ \mbox{The Complement Rule}\] }

\bigskip

{\Large \[ \mbox{www.Stats-Lab.com} \]}
\end{frame}
%--------------------------------------------------------------------------- %
\begin{frame}
\frametitle{The Complement Rule}
\Large
\begin{itemize}
\item The \textbf{\textit{Complement Rule}} is a very simple rule for working with probability distributions.
\item In this presentation, we will look at the Complement Rule for continuous probability distributions only.

\end{itemize}
\end{frame}

\begin{frame}
\frametitle{The Complement Rule}
\Large
\begin{itemize}

\item Remember, for continuous probability distributions, the probability of an \textbf{exact} value is extremely small, such that it is almost zero.
\[P(X = k) \approx 0\]
\item Therefore we neglect the equality components in expressions such as
$P(X \leq k)$ and $P(X \geq k)$.
\item In fact we can treat this two expressions as \textbf{\textit{complementary events}}.
\end{itemize}
\end{frame}



\begin{frame}
\frametitle{The Complement Rule}
\Large
\vspace{-2cm}
\[P(X \leq k) = 1- P(X \geq k) \]
\[P(X \geq k) = 1- P(X \leq k) \]
\begin{tabular}{|c|c|c|c|}
\hline Event &\phantom{s} Prob.\phantom{s} & Complement Event & \phantom{s} Prob.\phantom{s}\\ 
\hline $P(X \leq 100)$ & 0.65 &$P(X \geq 100)$  &  \\ 
\hline $P(Y \geq 80)$ & 0.40 & $P(8 \leq 80)$ &  \\ 
\hline 
\end{tabular} 

\end{frame}
\begin{frame}
\frametitle{The Complement Rule}
\Large
\vspace{-2cm}\begin{itemize}

\item To compute the probability of the complementary event, simple subtract the probability of the event from 1.

\[P(X \leq k) = 1- P(X \geq k) \]
\[P(X \geq k) = 1- P(X \leq k) \]
\end{itemize}
\end{frame}
\begin{frame}
\frametitle{The Complement Rule}
\Large
\begin{center}
\vspace{-2cm}
\[P(X \leq k) = 1- P(X \geq k) \]
\[P(X \geq k) = 1- P(X \leq k) \]
\begin{tabular}{|c|c|c|c|}
\hline Event &  & Complement  & \phantom{Event} \\ 
\hline $P(X \leq 100)$ & 0.65 &$P(X \geq 100)$  &  \\ 
\hline $P(Y \geq 80)$ & 0.40 & $P(Y \leq 80)$ &  \\ 
\hline 
\end{tabular} 
\end{center}
\end{frame}
\begin{frame}


\end{frame}

\end{document}