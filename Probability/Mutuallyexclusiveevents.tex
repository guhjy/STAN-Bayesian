\documentclass[IntroMain.tex]{subfiles} 
\begin{document}

%---------------------------------------------------------------------------------------------------%
\begin{frame}
\frametitle{Mutually Exclusive Events}
\Large
\begin{itemize}
\item Suppose you roll a 6 sided die. \item Let \textbf{A} be the event that the number is odd and \textbf{B} be the event that the number is even. 
\item Since the die is only rolled once, it is impossible for the number that lands face up is both odd and even. \item The events \textbf{A} and \textbf{B} are said to be mutually exclusive events. \item If two events cannot happen at the same time, they are said to be mutually exclusive.


\end{itemize}
\end{frame}
%---------------------------------------------------------------------------------------------------%
\begin{frame}
\frametitle{Mutually Exclusive Events}
\Large
\begin{itemize}
\item Two events are \textbf{mutually exclusive} if they cannot occur together. 
\item Another way of expressing mutually events is \textbf{disjoint} events.
\item If two events are mutually exclusive, then the probability of them both occurring at the same time is 0.
   Disjoint:  \[P(A \cap B) = 0\]
\end{itemize}
\end{frame}
%---------------------------------------------------------------------------------------------------%
\begin{frame}
\frametitle{Mutually Exclusive Events}
\Large
\begin{itemize}
\item If two events are mutually exclusive, then the probability of either occurring is the sum of the probabilities of each occurring.
\item Specific Addition Rule: Only valid when the events are mutually exclusive.
 \[P(A \cup B) = P(A) + P(B)\]
\end{itemize}   
\end{frame}

\begin{frame}
\frametitle{Independent Events}
\Large
\begin{itemize}
\item Two events are independent if the occurrence of one does not change the probability of the other occurring.
\item An example would be rolling a 2 on a die and flipping a head on a coin. Rolling the 2 does not affect the probability of flipping the head.
\item If events are independent, then the probability of them both occurring is the product of the probabilities of each occurring.
\end{itemize}
   \[P(A \cap B) = P(A) \times P(B)\]
\end{frame}
%---------------------------------------------------------------------------------------------------%   
\end{document}
