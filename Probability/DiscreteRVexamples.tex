\documentclass[IntroMain.tex]{subfiles} 
\begin{document}
	%------------------------------------------------------------%
%-----------------------------------------------------------------------------------------------------------%
\section*{Example}

\begin{frame}
	\frametitle{Introduction to Probability}
	\Large
	\textbf{Discrete Random Variables}
	
	A random variable is a numerical description of the outcome of an experiment.
	
	Random variables can be classified as discrete or continuous, depending on the numerical values they may take.
	
	A ranom variable that may assume any numerical value in an interval or collection of intervals is called a continuous random variable.
\end{frame}


%===============================================================%
\begin{frame}
	
	Question 3
	
	Suppose a fair coin is tossed six times. The number of heads which can occur with their respective
	probabilities are as follows:
	
	xi	0	1	2	3	4	5	6
	p(xi)	1/64	6/64	15/64	20/64	15/64	6/64	1/64
	
	a)	Compute the expected value (i.e. expected number of heads).
	b)	Compute the variance of the number of heads.
	
\end{frame}
%===============================================================%
%===============================================================%
\begin{frame}
	
	Question 4
	A player tosses two fair coins. He wins $\$2$ if two heads occur, and $\$1$ if one head occurs. On the other hand, he loses $\$3$ if no heads occur. 
	
	Find the expected value $E(X)$ of the game. Is the game fair? 
	
	% (The game is fair, favourable, or unfavourable to the player if E(X) = 0,E(X) > 0 or E(X) < 0 respectively)
	
\end{frame}
%===============================================================%
%===============================================================%
\begin{frame}
\frametitle{Discrete Random Variable : Example}
\Large
\vspace{-0.5cm}
For a particular Java assembler interface, the operand stack size has the
following probabilities:
\begin{center}
\begin{tabular}{|c||c|c|c|c|c|}
	\hline
	% after \\: \hline or \cline{col1-col2} \cline{col3-col4} ...
	Stack Size  & 0 & 1 & 2 & 3 & 4 \\ \hline
	Probability & .15 & .05 & .10 &.20 &.50\\
	\hline
\end{tabular}
\end{center}


\begin{itemize}

\item Calculate the expected stack size.
\item Calculate the variance of the stack size.

\end{itemize}
\end{frame}
\end{document}