\documentclass{beamer}

\usepackage{framed}
\usepackage{graphicx}
\usepackage{amsmath}
\usepackage{amssymb}

\begin{document}
\section{The binomial distribution} 

%http://www.wbs.eu.com/SharedFiles/Maths/statistics%201%20revision/introducing%20binomial.pdf

\begin{frame}
\frametitle{The Binomial Distribution}
The binomial distribution is a particular example of a probability distribution involving a discrete random variable. 
It is important that you can identify situations which can be modelled using the binomial distribution. 
\end{frame}
\end{document}
%-----------------------------------------------------------------------------------------------------------%
\begin{frame}
\begin{itemize}
\item There are n independent trials 
\item There are just two possible outcomes to each trial, \textbf{success} and \textbf{failure}, with fixed probabilities of $p$ and $q$ respectively, where $q = 1 – p$. 
\end{itemize}
\end{frame}

%-----------------------------------------------------------------------------------------------------------%
\begin{frame}
\frametitle{The Binomial Distribution}
The discrete random variable X is the number of successes in the n trials. 
X is modelled by the binomial distribution $B(n, p)$. 
You can write $X \sim B(n, p)$.
\end{frame}
%-----------------------------------------------------------------------------------------------------------%
\begin{frame}
\frametitle{Binomial Example 4} 
Using recent data provided by the low-cost arriving on time is estimated to be 0.9. 

On four different occasions I am taking a flight with Brianair. 
\begin{itemize}
\item[(i)] What is the probability that I arrive on time on all four flights? 
\item[(ii)] What is the probability that I arrive on time on exactly two occasions? 
\end{itemize}

\end{frame}
%-----------------------------------------------------------------------------------------------------------%
\end{document}