\documentclass[IntroMain.tex]{subfiles} 
\begin{document}
%=========================================================%
\begin{frame}
	\frametitle{Hypergeometric Distribution}
	\Large
\[	\mbox{Hypergeometric Distribution}\]
\end{frame}






%=========================================================================================%
\begin{frame}
\frametitle{Definition}

The following conditions characterize the hypergeometric distribution:
The result of each draw (the elements of the population being sampled) can be classified into one of two mutually exclusive categories (e.g. Pass/Fail or Female/Male or Employed/Unemployed).
The probability of a success changes on each draw, as each draw decreases the population (sampling without replacement from a finite population).
\end{frame}
%=========================================================================================%
\begin{frame}
A random variable X follows the hypergeometric distribution if its probability mass function (pmf) is given by[1]
\[ P(X = k) = \frac{\binom{K}{k} \binom{N - K}{n-k}}{\binom{N}{n}},\]
where
\begin{itemize}
\item N is the population size,
\item K is the number of success states in the population,
\item n is the number of draws,
\item k is the number of observed successes,
\item $\textstyle {a \choose b}$ is a binomial coefficient.
\end{itemize}
\end{frame}
%============================================================================================================ %
\begin{frame}
\frametitle{The Hypergeometric Distribution }
	
	When sampling is done without replacement of each sampled item taken from a finite population of items, the
	Bernoulli process does not apply because there is a systematic change in the probability of success as items are
	removed from the population. 
	
\end{frame}
%=========================================================================================================== %
\begin{frame}
\frametitle{The Hypergeometric Distribution }	
	
	\begin{itemize}
	\item	When sampling without replacement is used in a situation that would otherwise
		qualify as a Bernoulli process, the hypergeometric distribution is the appropriate discrete probability distribution.
	\item	Given that X is the designated number of successes, N is the total number of items in the population, T is
		the total number of successes included in the population, and n is the number of items in the sample, the formula
		for determining hypergeometric probabilities is
	\end{itemize}

	
\end{frame}

%=========================================================================================================== %
\end{document}
