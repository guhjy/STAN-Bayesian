\documentclass[IntroMain.tex]{subfiles} 
\begin{document}
%=========================================================%
\begin{frame}
	\frametitle{Hypergeometric Distribution}
	\Large
\[	\mbox{Hypergeometric Distribution}\]
\end{frame}
%============================================================================================================ %
\begin{frame}
\frametitle{The Hypergeometric Distribution }
	
	When sampling is done without replacement of each sampled item taken from a finite population of items, the
	Bernoulli process does not apply because there is a systematic change in the probability of success as items are
	removed from the population. 
	
\end{frame}
%=========================================================================================================== %
\begin{frame}
\frametitle{The Hypergeometric Distribution }	
	
	\begin{itemize}
	\item	When sampling without replacement is used in a situation that would otherwise
		qualify as a Bernoulli process, the hypergeometric distribution is the appropriate discrete probability distribution.
	\item	Given that X is the designated number of successes, N is the total number of items in the population, T is
		the total number of successes included in the population, and n is the number of items in the sample, the formula
		for determining hypergeometric probabilities is
	\end{itemize}

	
\end{frame}

%=========================================================================================================== %
\end{document}
