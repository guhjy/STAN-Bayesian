\documentclass[]{article}
\usepackage{amsmath}
\usepackage{graphics}
\usepackage{graphicx}
\begin{document}
%http://www.stats.gla.ac.uk/steps/glossary/hypothesis_testing.html
\section*{Geometric Distribution}
Geometric distributions model (some) discrete random variables. Typically, a Geometric random variable is the number of trials required to obtain the first failure, for example, the number of tosses of a coin untill the first 'tail' is obtained, or a process where components from a production line are tested, in turn, until the first defective item is found.

A discrete random variable X is said to follow a Geometric distribution with parameter p, written $X \sim Ge(p)$, if it has probability distribution
\[P(X=x) = p^{x-1}(1-p)^x\]
where
\begin{itemize}
\item $x = 1, 2, 3, \ldots$
\item p = success probability; $0 < p < 1$
\end{itemize}
The trials must meet the following requirements:

\begin{itemize}
\item[(i)] the total number of trials is potentially infinite;
there are just two outcomes of each trial; success and failure;
\item[(ii)] the outcomes of all the trials are statistically independent;
\item[(iii)] all the trials have the same probability of success.
\end{itemize}
The Geometric distribution has expected value and variance  \[E(X)= 1/(1-p)\] \[V(X)=p/{(1-p)^2}\].

The Geometric distribution is related to the Binomial distribution in that both are based on independent trials in which the probability of success is constant and equal to $p$. 

However, a Geometric random variable is the number of trials until the first failure, whereas a Binomial random variable is the number of successes in n trials.

\end{document}

\end{document}