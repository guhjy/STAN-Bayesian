\frame{
\frametitle{Sampling}
The major use of statistics is to use information from a \textit{\textbf{sample}} to infer something about a \textit{\textbf{population}}.
\begin{itemize}


\item A \textit{\textbf{population}} is a collection of data whose properties are analyzed. The population is the complete collection to be studied, it contains all subjects of interest.

\item A \textit{\textbf{sample}} is a part of the population of interest, a sub-collection selected from a population.

\item
A \textit{\textbf{parameter}} is a numerical measurement that describes a characteristic of a population, while a \textit{\textbf{sample statistic}} is a numerical measurement that describes a characteristic of a sample.

\item In general, we will use a statistic to infer something about a parameter. 
\end{itemize}
}


\frame{
\frametitle{Sampling without replacement}
\begin{itemize}
\item Sampling is said to be ``without replacement" when a unit is selected at random from the population and it is not returned to the main lot. \item The first unit is selected out of a population of size $N$ and the second unit is selected out of the remaining population of  $N-1$ units and so on.
    \item For example, if you draw one card out of a deck of 52, there are only 51 cards left to draw from if you are selecting a second card.
\end{itemize}
}

\frame{
\frametitle{Sampling without replacement}
A lot of 100 semiconductor chips contains 20 that are defective.
Two chips are selected at random, without replacement from the lot.
\begin{itemize}
\item What is the probability that the first one is defective? \\(Answer : 20/100 , i.e 0.20)
\item What is the probability that the second one is defective given that the first one was defective? \\(Answer: 19/99)
\item What is the probability that the second one is defective given that the first one was not defective? \\(Answer: 20/99)
\end{itemize}
}

\frame{
\frametitle{Sampling With Replacement }

Sampling is called ``with replacement" when a unit selected at random from the population is returned to the population and then a second element is selected at random. Whenever a unit is selected, the population contains all the same units.
\begin{itemize}
\item What is the probability of guessing a PIN number for an ATM card at the first attempt.

\item Importantly a digit can be used twice, or more, in PIN codes.

\item For example $1337$ is a valid pin number, where $3$ appears twice.

\item
We have a one-in-ten chance of picking the first digit correctly, a one-in-ten chance of the guessing the second, and so on.

\item All of these events are independent, so the probability of guess the correct PIN is $0.1 \times 0.1 \times 0.1 \times 0.1 = 0.0001$
\end{itemize}
}



\frame{
	\frametitle{Sampling without replacement}
	\begin{itemize}
		\item Sampling is said to be ``without replacement" when a unit is selected at random from the population and it is not returned to the main lot. \item The first unit is selected out of a population of size $N$ and the second unit is selected out of the remaining population of  $N-1$ units and so on.
		\item For example, if you draw one card out of a deck of 52, there are only 51 cards left to draw from if you are selecting a second card.
	\end{itemize}
}

\frame{
	\frametitle{Sampling without replacement}
	A lot of 100 semiconductor chips contains 20 that are defective.
	Two chips are selected at random, without replacement from the lot.
	\begin{itemize}
		\item What is the probability that the first one is defective? \\(Answer : 20/100 , i.e 0.20)
		\item What is the probability that the second one is defective given that the first one was defective? \\(Answer: 19/99)
		\item What is the probability that the second one is defective given that the first one was not defective? \\(Answer: 20/99)
	\end{itemize}
}

\frame{
	\frametitle{Sampling With Replacement }
	
	Sampling is called ``with replacement" when a unit selected at random from the population is returned to the population and then a second element is selected at random. Whenever a unit is selected, the population contains all the same units.
	\begin{itemize}
		\item What is the probability of guessing a PIN number for an ATM card at the first attempt.
		
		\item Importantly a digit can be used twice, or more, in PIN codes.
		
		\item For example $1337$ is a valid pin number, where $3$ appears twice.
		
		\item
		We have a one-in-ten chance of picking the first digit correctly, a one-in-ten chance of the guessing the second, and so on.
		
		\item All of these events are independent, so the probability of guess the correct PIN is $0.1 \times 0.1 \times 0.1 \times 0.1 = 0.0001$
	\end{itemize}
}