%------------------------------------------------------------%

\frame{
\frametitle{Probability}
\begin{itemize}
\item Probability theory is the mathematical study of randomness. A
probability model of a random experiment is defined by assigning
probabilities to all the different outcomes.
\item Probability is a numerical measure of the likelihood that an event will
occur. Thus, probabilities can be used as measures of degree of
uncertainty associated with outcomes of an experiment.
Probability values are always assigned on a scale from 0 to 1.
\item A probability of 0 means that the event is impossible, while
a probability near 0 means that it is highly unlikely to occur.
\item Similarly an event with probability 1 is certain to occur, whereas an
event with a probability near to 1 is very likely to occur.
\end{itemize}

}
%--------------------------------------------------------------------------------%
\frame{
\frametitle{Experiments and Outcomes}
\begin{itemize}
\item In the study of probability any process of observation is referred to as an
experiment.
\item The results of an experiment (or other situation involving uncertainty)
are called the outcomes of the experiment.
\item An experiment is called a random experiment if the outcome can not be
predicted.
\item Typical examples of a random experiment are
\begin{itemize}
\item a role of a die,
\item a toss of a coin,
\item drawing a card from a deck.
\end{itemize}If the experiment is yet to be performed we refer to ‘possible outcomes’
or ‘possibilities’ for short. If the experiment has been performed, we
refer to ‘realized outcomes’ or ‘realizations’.
\end{itemize}
}

%--------------------------------------------------------------------------------%
\frame{
\frametitle{Sample Spaces and Events}

\begin{itemize}
\item The set of all possible outcomes of a probability experiment is called a
\textbf{\emph{sample space}}, which is usually denoted by \textbf{\emph{S}}.
\item The sample space is an exhaustive list of all the possible outcomes of an
experiment. We call individual elements of this list \textbf{\emph{sample points}}.
\item Each possible outcome is represented by one and only one sample point
in the sample space.
\end{itemize}
}

%--------------------------------------------------------------------------------%
\frame{
\frametitle{Sample Spaces: Examples}
For each of the following experiments, write out the sample space.
\begin{itemize}
\item Experiment: Rolling a die once
\begin{itemize}
\item Sample space $S = \{1,2,3,4,5,6\}$
\end{itemize}
\item Experiment: Tossing a coin
\begin{itemize}
\item Sample space $S = \{ Heads , Tails\}$
\end{itemize}
\item Experiment: Measuring a randomly selected person’s height (cms)
\begin{itemize}
\item Sample space $S =$ The set of all possible real numbers.
\end{itemize}
\end{itemize}
}
%--------------------------------------------------------------------------------%
\frame{
\frametitle{Events}

\begin{itemize} \item An event is a specific outcome, or any collection of outcomes of an
experiment.
\item Formally, any subset of the sample space is an event.
\item Any event which consists of a single outcome in the sample space is
called an \textbf{\emph{elementary}} or \textbf{\emph{simple event}}.
\item Events which consist of more than one outcome are called \textbf{\emph{compound
events.}}
\item For example, an elementary event associated with the die example could
be the ``die shows 3".
\item An compound event associated with the die example could be the ``die
shows an even number".
\end{itemize}
}
%--------------------------------------------------------------------------------%
\frame{
\frametitle{The Complement Event}

\begin{itemize} 

\item The complement of an event $A$ is the set of all outcomes in the sample
space that are not included in the outcomes of event $A$.
\item We call the complement event of $A$ as $A^c$.
\item The complement event of a die throw resulting in an even number is the
die throwing an odd number.
\item Question: if there is a $40\%$ chance of a randomly selected student being male, what is the probability of the selected student being female?
\end{itemize}
}

%--------------------------------------------------------------------------------%
\frame{
\frametitle{Set Theory : Union and Intersection}

Set theory is used to represent relationships among events.\\ \bigskip

\noindent \textbf{Union of two events:}\\
The union of events A and B is the event containing all the sample points
belonging to A or B or both. This is denoted $A\cup B$, (pronounce as ``A union
B").\\ \bigskip
\noindent \textbf{Intersection of two events:}\\
The intersection of events A and B is the event containing all the sample
points common to both A and B. This is denoted $A\cap B$, (pronounce as ``A intersection
B").
}

%--------------------------------------------------------------------------------%
\frame{
\frametitle{More Set Theory}

In general, if A and B are two events in the sample space S, then
\begin{itemize} 
\item $A \subseteq B$ (A is a subset of B) = `if A occurs, so does B’
\item $\varnothing$ (the empty set) = an impossible event
\item $S$ (the sample space) = an event that is certain to occur
\end{itemize}
}

%--------------------------------------------------------------------------------%
\frame{
\frametitle{Examples of Events}

Consider the experiment of rolling a die once. From before, the sample space
is given as $S = \{ 1,2,3,4,5,6\}$. The following are examples of possible events.
\begin{itemize} 
\item A = score $< 4$ = $\{ 1,2,3\}$.
\item B = `score is even' = $\{ 2,4,6\}$.
\item C = `score is 7' = 0
\item $A\cup B$ = `the score is $< 4$  or even or both' = $\{ 1,2,3,4,6\}$
\item $A\cap B$ = `the score is $< 4$  and even’ = $ \{ 2 \}$
\item $A^c$ =`event A does not occur' = $ \{ 4,5,6\}$
\end{itemize}
}
