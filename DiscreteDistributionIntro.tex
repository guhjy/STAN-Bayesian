\documentclass[IntroMain.tex]{subfiles} 
\begin{document}
	%=========================================================%
	\begin{frame}
		\frametitle{Geometric Distribution}
		\Large
		\[	\mbox{Geometric Distribution}\]
	\end{frame}


%--------------------------------------------------------------------------------------%
\frame{
	\frametitle{Discrete Probability Distributions}
	
	\begin{itemize}
		
		\item Over the next set of lectures, we are now going to look at two important discrete probability distributions
		
		\item The first is the \textbf{\emph{binomial}} probability distribution.
		
		\item The second is the Poisson probability distribution.
		
		\item In \texttt{R}, calculations are performed using the \texttt{binom} family of functions and \texttt{pois} family of functions respectively.
	\end{itemize}
	
}




%-------------------------------------------------------------%
\frame{
	\begin{itemize}
		\item Now consider an experiment with only two outcomes. Independent repeated trials of such an experiment are
		called Bernoulli trials, named after the Swiss mathematician Jacob Bernoulli (1654–1705). \item The term \textbf{\emph{independent
				trials}} means that the outcome of any trial does not depend on the previous outcomes (such as tossing a coin).
		\item We will call one of the outcomes the ``success" and the other outcome the ``failure".
	\end{itemize}
}

%-------------------------------------------------------------%
\frame{
	\begin{itemize}
		\item
		Let $p$ denote the probability of success in a Bernoulli trial, and so $q = 1 - p$ is the probability of failure.
		A binomial experiment consists of a fixed number of Bernoulli trials. \item A binomial experiment with $n$ trials and
		probability $p$ of success will be denoted by
		\[B(n, p)\]
	\end{itemize}
}
%-------------------------------------------------------------%

%---------------------------------------------------------------------------%
\frame{
	\frametitle{Probability Mass Function}
	\begin{itemize} \item a probability mass function (pmf) is a function that gives the probability that a discrete random variable is exactly equal to some value. \item The probability mass function is often the primary means of defining a discrete probability distribution \end{itemize}
}
%------------------------------------------------------------------%
\frame{
	Thirty-eight students took the test. The X-axis shows various intervals of scores (the interval labeled 35 includes any score from 32.5 to 37.5). The Y-axis shows the number of students scoring in the interval or below the interval.
	
	\textbf{\emph{cumulative frequency distribution}}A  can show either the actual frequencies at or below each interval (as shown here) or the percentage of the scores at or below each interval. The plot can be a histogram as shown here or a polygon.
}




\end{document}



%---------------------------------------------------------------------------------------------------------------%
