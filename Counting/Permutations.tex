\documentclass[IntroMain.tex]{subfiles} 
\begin{document}




\begin{frame}
	\frametitle{Permutations}
\large
\begin{itemize}
\item The number of permutations of n objects is the number of ways in which the objects can be arranged in
terms of order: \item
Permutations of n objects : \[n! = (n) \times (n - 1) \times (n-2) \ldots  \times 2 \times 1 \]
\item

The symbol n! is read ``n factorial". 
\item In permutations and combinations problems, n is always positive.
Also, note that by definition $0! = 1$ in mathematics.
	\end{itemize}
\end{frame}

\section{Combinations}
\begin{frame}
\frametitle{Combinations}
\begin{itemize}
\item In the case of permutations, the order in which the objects are arranged is important.
\item In the case of
combinations, we are concerned with the number of different groupings of objects that can occur without regard
to their order. 
\item Therefore, an interest in combinations always concerns the number of different subgroups that
can be taken from n objects. The number of combinations of n objects taken r at a time is
\end{itemize}	
%=========================================================================== %	
\end{frame}
\begin{frame}
\frametitle{Permutations}
\Large
\vspace{-1.0cm}
Suppose a four letter code is made from the letters \textbf{\textit{\{a,b,c,d,e\}}}, where repetitions are allowed and the order of the letters in the code is significant\\ \bigskip For example
\textbf{\textit{a,a,e,c}} is a different code to \textbf{\textit{a,c,e,a}}.
\end{frame}


\begin{frame}

\frametitle{Permutations}
\Large
\begin{itemize}
\item Let $\mathcal{U}$ be the set of all such codes.
\item Let $\mathcal{V}$ be the set of all such codes beginning with a vowel.
\item Let $\mathcal{P}$ be the set of all such codes which are palindromic.
\end{itemize} 
\bigskip
(A palindromic code is a string of letters which read the same backwards as forwards, for example \textbf{\textit{a,e,c,e,a}} is a 5 letter palindromic code.)\\ \bigskip

\end{frame}

\begin{frame}

\frametitle{Permutations}
\Large
How many elements are there in the set $\mathcal{U}$?
\begin{center}
\begin{tabular}{|c|c|c|c|}
\hline (i) &  (ii) &  (iii) &  (iv) \\ 
\hline {\color{white}More space} &{\color{white}More space}  & {\color{white}More space} &{\color{white}More space}  \\ 
 {\color{white}More space} &{\color{white}More space}  & {\color{white}More space} &{\color{white}More space}  \\ 
\hline 
\end{tabular} 
\end{center}
\end{frame}

\begin{frame}

\frametitle{Permutations}
\Large
How many elements are there in the set $\mathcal{V}$?
\begin{center}
\begin{tabular}{|c|c|c|c|}
\hline (i) &  (ii) &  (iii) &  (iv) \\ 
\hline {\color{white}More space} &{\color{white}More space}  & {\color{white}More space} &{\color{white}More space}  \\ 
 {\color{white}More space} &{\color{white}More space}  & {\color{white}More space} &{\color{white}More space}  \\ 
\hline 
\end{tabular} 
\end{center}
\end{frame}

\begin{frame}

\frametitle{Permutations}
\Large
How many elements are there in the set $\mathcal{P}$?
\begin{center}
\begin{tabular}{|c|c|c|c|}
\hline (i) &  (ii) &  (iii) &  (iv) \\ 
\hline {\color{white}More space} &{\color{white}More space}  & {\color{white}More space} &{\color{white}More space}  \\ 
 {\color{white}More space} &{\color{white}More space}  & {\color{white}More space} &{\color{white}More space}  \\ 
\hline 
\end{tabular} 
\end{center}
\end{frame}
\begin{frame}

\frametitle{Permutations}
\Large
How many elements are there in the sets $\mathcal{V}$ and $\mathcal{P}$?
\begin{center}
\begin{tabular}{|c|c|c|c|}
\hline (i) &  (ii) &  (iii) &  (iv) \\ 
\hline {\color{white}More space} &{\color{white}More space}  & {\color{white}More space} &{\color{white}More space}  \\ 
 {\color{white}More space} &{\color{white}More space}  & {\color{white}More space} &{\color{white}More space}  \\ 
\hline 
\end{tabular} 
\end{center}
\end{frame}
\begin{frame}
Empty
\end{frame}
\end{document}
