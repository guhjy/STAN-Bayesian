\documentclass{beamer}

\usepackage{amsmath}
\usepackage{amssymb}
\usepackage{framed}
\usepackage{multicol}

\begin{document}

\begin{frame}
\Huge
\[\mbox{Introduction to Probability}\]
\LARGE
\[\mbox{Permutations with Constraints}\]

\Large
\[\mbox{kobriendublin.wordpress.com}\]
\[\mbox{Twitter: @StatsLabDublin}\]

\end{frame}

\begin{frame}
\frametitle{Permutations with Constraints}
\Large
\vspace{-2cm}
How many different four digit numbers greater than 5000 can be formed from the digits \[2,4,5,8,9\] if each digit can only be used once in any given number.
\\

 

\end{frame}

\begin{frame}
\frametitle{Permutations with Constraints}
\Large
\vspace{-3cm}
How many of these four digit numbers are odd, given they are greater than 5000?
\[2,4,5,8,9\]\\

\end{frame}


\begin{frame}
\frametitle{Permutations with Constraints}
\Large


\end{frame}

 
 
 
%http://www.mathsireland.com/LCHGeneralNotes/PermCombProb/5_5_Prob_MultAnd/Q_5_5_Prob_MultAnd.html
\begin{frame}
\Huge
\[\mbox{Introduction to Probability}\]
\LARGE
\[\mbox{Calculations using the Choose Operator}\]

\Large
\[\mbox{kobriendublin.wordpress.com}\]
\[\mbox{Twitter: @StatsLabDublin}\]

\end{frame}

\begin{frame}
\frametitle{Choose Operator}
\Large
For the positive integer $n$ and non-negative integer $k$ ( with $k\leq n$), the choose operater is calculated as follows:

\[ {n \choose k} = \frac{n!}{k! \times (n-k)!} \]

\end{frame}

\begin{frame}
\frametitle{Choose Operator}
\Large
\vspace{-1.5cm}
Evaluate the following:
\huge
\begin{multicols}{3}
    \begin{enumerate}
    \item ${5 \choose 2}$
    \item ${5 \choose 0}$
    \item ${6 \choose 3}$
    \item ${6 \choose 6}$
    \item ${10 \choose 1}$
    \item ${10 \choose 9}$
    \end{enumerate}        
  \end{multicols}
\end{frame}
%------------------------------------------ %
\begin{frame}
\frametitle{Choose Operator}
\large
\vspace{-3cm}
\textbf{Part 1}
\huge
\[{5 \choose 2}\]

\end{frame}
%------------------------------------------ %
\begin{frame}
\frametitle{Choose Operator}
\large
\vspace{-3cm}
\textbf{Part 2}
\huge
\[{5 \choose 0}\]

\end{frame}
%------------------------------------------ %
\begin{frame}
\frametitle{Choose Operator}
\large
\vspace{-3cm}
\textbf{Part 3}
\huge
\[{6 \choose 3}\]

\end{frame}
%------------------------------------------ %
\begin{frame}
\frametitle{Choose Operator}
\large
\vspace{-3cm}
\textbf{Part 4}
\huge
\[{6 \choose 6}\]

\end{frame}
%------------------------------------------ %
\begin{frame}
\frametitle{Choose Operator}
\large
\vspace{-3cm}
\textbf{Part 5}
\huge
\[{10 \choose 1}\]

\end{frame}
%------------------------------------------ %
\begin{frame}
\frametitle{Choose Operator}
\large
\vspace{-3cm}
\textbf{Part 6}
\huge
\[{10 \choose 9}\]

\end{frame}
%------------------------------------------ %
\begin{frame}

\end{frame}
%------------------------------------------ %
\begin{frame}
\frametitle{Counting Sets with Venn Diagrams}
\Large
\vspace{-2cm}
%----------------------------------------%
\begin{itemize}
\item The Venn Diagram shows the number of elements in each subset of set $S$.
\item If $P(A) = 3/10$ and $P(B) = 1/2$, find the values of $x$ and $y$
\end{itemize}
%----------------------------------------%
%Venn Diagrams
\end{frame}



\begin{frame}
\frametitle{Counting Sets with Venn Diagrams}
\Large
\vspace{-1cm}
%----------------------------------------%
\begin{itemize}
\item The total number of items in the data set is $x+y+5$
\item There are $x+1$ items in Area $A$
\item There are $x+y$ items in Area $B$
\item We can say
\[ P(A) = \frac{3}{10} = \frac{x+1}{x+y+5}\]
\[ P(B) = \frac{1}{2} = \frac{x+y}{x+y+5} \]
\end{itemize}
%----------------------------------------%
\end{frame}



\begin{frame}
\frametitle{Counting Sets with Venn Diagrams}
\Large
\vspace{-2.5cm}
\textbf{Cross Multiplication}
\[ P(A) = \frac{3}{10} = \frac{x+1}{x+y+5}\]

\end{frame}


\begin{frame}
\frametitle{Counting Sets with Venn Diagrams}
\Large
\vspace{-2.5cm}
\textbf{Cross Multiplication}
\[ P(B) = \frac{1}{2} = \frac{x+y}{x+y+5} \]

\end{frame}


\begin{frame}
\frametitle{Counting Sets with Venn Diagrams}
\Large
\vspace{-2.8cm}
\textbf{Simultaneous Equations}
\begin{itemize}
\item[1)] $7x-3y=5$
\item[2)] $x+y=5$
\end{itemize}
\end{frame}

\begin{frame}
\frametitle{Counting Sets with Venn Diagrams}
\Large
\vspace{-2.8cm}
\textbf{Simultaneous Equations}
\begin{itemize}
\item $7x-3y=5$
\item $x+y=5$
\end{itemize}
\end{frame}
\end{document}
