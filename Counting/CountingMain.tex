\documentclass[IntroMain.tex]{subfiles} 
\begin{document}
	
	%=========================================================%
	\begin{frame}
		\frametitle{Introduction to Probability}
		\Large
		\textbf{Techniques for Counting}
		
		\begin{itemize}
			\item Combinations
			\item Permutations
			\item Permutations with constraints
		\end{itemize}
		
	\end{frame}
	%=========================================================%
	\begin{frame}
		\frametitle{Introduction to Probability}
		\Large
		
		\textbf{Permuations of subsets}
		
		The number of permutations of subsets of $k$ elements selected from a set of $n$ different elements is
		
		\[P(n,r) = \frac{n!}{(n-k)!}  \]
	\end{frame}
	%=========================================================%
	\begin{frame}
		\frametitle{Introduction to Probability}
		\Large
		
		\textbf{Combinations of subsets}
		
		The number of combinations that can be selected from $n$ items is
		
		\[ {n \choose k} = \frac{n!}{k! \times (n-k)!}  \]
	\end{frame}
	%=========================================================%
	
	%--------------------------------------------------------%
	\frame{
		\large
		\frametitle{Counting Problems}
		\begin{itemize}
			\item Sampling without replacement.
			\item Factorials
			\item Permutations
			\item Combinations
		\end{itemize}
		
	}
	
	
	
	\frame{
		\frametitle{Sampling without replacement}
		\begin{itemize}
			\item Sampling is said to be ``without replacement" when a unit is selected at random from the population and it is not returned to the main lot. \item The first unit is selected out of a population of size $N$ and the second unit is selected out of the remaining population of  $N-1$ units and so on.
			\item For example, if you draw one card out of a deck of 52, there are only 51 cards left to draw from if you are selecting a second card.
		\end{itemize}
	}
	
	\frame{
		\frametitle{Sampling without replacement}
		A lot of 100 semiconductor chips contains 20 that are defective.
		Two chips are selected at random, without replacement from the lot.
		\begin{itemize}
			\item What is the probability that the first one is defective? \\(Answer : 20/100 , i.e 0.20)
			\item What is the probability that the second one is defective given that the first one was defective? \\(Answer: 19/99)
			\item What is the probability that the second one is defective given that the first one was not defective? \\(Answer: 20/99)
		\end{itemize}
	}
	
	\frame{
		\frametitle{Sampling With Replacement }
		
		Sampling is called ``with replacement" when a unit selected at random from the population is returned to the population and then a second element is selected at random. Whenever a unit is selected, the population contains all the same units.
		\begin{itemize}
			\item What is the probability of guessing a PIN number for an ATM card at the first attempt.
			
			\item Importantly a digit can be used twice, or more, in PIN codes.
			
			\item For example $1337$ is a valid pin number, where $3$ appears twice.
			
			\item
			We have a one-in-ten chance of picking the first digit correctly, a one-in-ten chance of the guessing the second, and so on.
			
			\item All of these events are independent, so the probability of guess the correct PIN is $0.1 \times 0.1 \times 0.1 \times 0.1 = 0.0001$
		\end{itemize}
	}
	%--------------------------------------------------------%
	\section{ Combinations and Permutations }
	
	%--------------------------------------------------------%
	\frame{
		\frametitle{Factorials Numbers}
		
		A factorial is a positive whole number, based on a number $n$ , and which is written as $``n!"$. The factorial $n!$ is defined as follows:
		
		\[n!  =n \times (n-1) \times (n-2) \times \ldots \times 2 \times 1 \]
		
		Remark $n!  =n \times (n-1)!$\\ \bigskip
		
		\textbf{ Example: }
		
		\begin{itemize}
			\item $3!  = 3 \times 2  \times 1 = 6 $
			
			\item $4!  = 4 \times 3! = 4 \times 3 \times 2 \times 1 = 24$
		\end{itemize}
		Remark $0! = 1$ not $0$.
		
		
	}
	
	%--------------------------------------------------------%
	\frame{
		\frametitle{Permutations and Combinations}
		
		
		Often we are concerned with computing the number of ways of selecting and arranging groups of items. \begin{itemize} \item  A \textbf{\emph{combination}} describes the selection of items from a larger group of items.  \item A \textbf{\emph{permutation}} is a combination that is arranged in a particular way.
		\end{itemize}
		
		\bigskip
		\begin{itemize}
			\item Suppose we have items A,B,C and D to choose two items from.
			\item AB is one possible selection, BD is another. AB and BD are both combinations.
			\item More importantly, AB is one combination, for which there are two distinct permutations: AB and BA.
		\end{itemize}
	}
	
	%--------------------------------------------------------%
	\frame{
		\frametitle{Combinations}
		
		\textbf{Combinations: }
		The number of ways of selecting $k$ objects from $n$ unique objects is:
		
		\[ ^n C_k = {n!  \over k! \times (n-k)!} \]
		
		In some texts, the notation for finding the number of possible combination is written
		
		\[ ^n C_k =  {n \choose k} \]
		
	}
	
	%--------------------------------------------------------%
	\frame{
		\frametitle{Example of Combinations}
		How many ways are there of selecting two items from possible 5?
		
		\[ ^5 C_2   \left( \mbox{ also }  {5 \choose 2}  \right) =  {5!  \over 2! \times 3!} =  {5 \times 4 \times 3!  \over 2 \times 1 \times 3!} = 10  \]
		
		\bigskip
		Discuss how combinations can be used to compute the number of rugby matches for each group in the Rugby World Cup.
		
	}
	%--------------------------------------------------------%
	\frame{
		\frametitle{The Permutation Formula}
		The number of different permutations of r items from n unique items is written as $^n P_k$
		
		
		\[ ^n P_k = \frac{n!}{(n-k)!}\]
	}
	
	%--------------------------------------------------------%
	\frame{
		\frametitle{Permutations}
		\textbf{Example:}
		How many ways are there of arranging 3 different jobs, between 5 workers, where each worker can only do one job?
		
		
		\[ ^5 P_3 = \frac{5!}{(5-3)!}  = {5! \over 2!} = 60\]
		
	}
	
	
	
	%--------------------------------------------------------%
	\frame{
		\frametitle{Example of Combinations}
		
		A committee of 4 must be chosen from 3 females and 4 males.
		
		\begin{itemize}
			\item In how many ways can the committee be chosen.
			\item In how many cans 2 males and 2 females be chosen.
			\item Compute the probability of a committee of 2 males and 2 females are chosen.
			\item Compute the probability of at least two females.
		\end{itemize}
	}
	
	%--------------------------------------------------------%
	\frame{
		\frametitle{Example of Combinations}
		
		\textbf{Part 1}
		
		We need to choose 4 people from 7:
		
		This can be done in
		
		\[
		^7 C_4  = {7!  \over 4! \times 3!} =  {7 \times 6 \times 5 \times 4!  \over 4! \times 3!} = 35 \mbox{ ways.}
		\]
		
		
		\textbf{Part 2}
		
		With 4 men to choose from, 2 men can be selected in \[
		^4 C_2  = {4!  \over 2! \times 2!} =  {4 \times 3 \times 2!  \over 2! \times 2!} = 6\mbox{ ways.}
		\]
		
		Similarly 2 women can be selected from 3 in
		\[
		^3 C_2  = {3!  \over 2! \times1!} =  {3 \times 2!  \over 2! \times 1!} = 3\mbox{ ways.}
		\]
		
	}
	
	%=============================================================================================== %
	\frame{
		\frametitle{Example of Combinations}
		
		\textbf{Part 2}
		
		Thus a committee of 2 men and 2 women can be selected in $ 6 \times 3  = 18 $ ways.\\
		\bigskip
		\textbf{Part 3}
		
		The probability of two men and two women on a committee is
		\[ {\mbox{Number of ways of selecting 2 men and 2 women} \over \mbox{Number of ways of selecting 4 from 7}} = {18 \over 35 }\]
		
	}
	%--------------------------------------------------------%
	\frame{
		\frametitle{Example of Combinations}
		
		\textbf{Part 4}
		\begin{itemize}
			\item The probability of at least two females is the probability of 2 females or 3 females being selected.
			\item We can use the addition rule, noting that these are two mutually exclusive events.
			\item From before we know that probability of 2 females being selected is 18/35.
		\end{itemize}
		
	}
	%--------------------------------------------------------%
	\frame{
		\frametitle{Example of Combinations}
		\Large
		\textbf{Part 4}
		\begin{itemize}
			\item We have to compute the number of ways of selecting 1 male from 4 (4 ways) and the number of ways of selecting three females from 2 ( only 1 way)
			\item The probability of selecting three females is therefore ${4 \times 1 \over 35} = 4/35$
			\item So using the addition rule
			\[ Pr(\mbox{ at least 2 females }) = Pr(\mbox{ 2 females }) + Pr(\mbox{ 3 females }) \]
			\[ Pr(\mbox{ at least 2 females })  = 18/35 + 4/35 = 22/35 \]
		\end{itemize}
		
	}
	
	
	\begin{frame}
		\frametitle{Permutations with Constraints}
		\Large
		\vspace{-2cm}
		How many different four digit numbers greater than 5000 can be formed from the digits \[2,4,5,8,9\] if each digit can only be used once in any given number.
		\\
		
		
		
	\end{frame}
	
	\begin{frame}
		\frametitle{Permutations with Constraints}
		\Large
		\vspace{-3cm}
		How many of these four digit numbers are odd, given they are greater than 5000?
		\[2,4,5,8,9\]\\
		
	\end{frame}
	
	
	\begin{frame}
		\frametitle{Permutations with Constraints}
		\Large
		
		
	\end{frame}
	
	
	
	
	%http://www.mathsireland.com/LCHGeneralNotes/PermCombProb/5_5_Prob_MultAnd/Q_5_5_Prob_MultAnd.html
	\begin{frame}
		\Huge
		\[\mbox{Introduction to Probability}\]
		\LARGE
		\[\mbox{Calculations using the Choose Operator}\]
		
		\Large
		\[\mbox{kobriendublin.wordpress.com}\]
		\[\mbox{Twitter: @StatsLabDublin}\]
		
	\end{frame}
	
	\begin{frame}
		\frametitle{Choose Operator}
		\Large
		For the positive integer $n$ and non-negative integer $k$ ( with $k\leq n$), the choose operater is calculated as follows:
		
		\[ {n \choose k} = \frac{n!}{k! \times (n-k)!} \]
		
	\end{frame}
	
	\begin{frame}
		\frametitle{Choose Operator}
		\Large
		\vspace{-1.5cm}
		Evaluate the following:
		\huge
		\begin{multicols}{3}
			\begin{enumerate}
				\item ${5 \choose 2}$
				\item ${5 \choose 0}$
				\item ${6 \choose 3}$
				\item ${6 \choose 6}$
				\item ${10 \choose 1}$
				\item ${10 \choose 9}$
			\end{enumerate}        
		\end{multicols}
	\end{frame}
	%------------------------------------------ %
	\begin{frame}
		\frametitle{Choose Operator}
		\large
		\vspace{-3cm}
		\textbf{Part 1}
		\huge
		\[{5 \choose 2}\]
		
	\end{frame}
	%------------------------------------------ %
	\begin{frame}
		\frametitle{Choose Operator}
		\large
		\vspace{-3cm}
		\textbf{Part 2}
		\huge
		\[{5 \choose 0}\]
		
	\end{frame}
	%------------------------------------------ %
	\begin{frame}
		\frametitle{Choose Operator}
		\large
		\vspace{-3cm}
		\textbf{Part 3}
		\huge
		\[{6 \choose 3}\]
		
	\end{frame}
	%------------------------------------------ %
	\begin{frame}
		\frametitle{Choose Operator}
		\large
		\vspace{-3cm}
		\textbf{Part 4}
		\huge
		\[{6 \choose 6}\]
		
	\end{frame}
	%------------------------------------------ %
	\begin{frame}
		\frametitle{Choose Operator}
		\large
		\vspace{-3cm}
		\textbf{Part 5}
		\huge
		\[{10 \choose 1}\]
		
	\end{frame}
	%------------------------------------------ %
	\begin{frame}
		\frametitle{Choose Operator}
		\large
		\vspace{-3cm}
		\textbf{Part 6}
		\huge
		\[{10 \choose 9}\]
		
	\end{frame}
	%------------------------------------------ %
	\begin{frame}
		
	\end{frame}
	%------------------------------------------ %
	\begin{frame}
		\frametitle{Counting Sets with Venn Diagrams}
		\Large
		\vspace{-2cm}
		%----------------------------------------%
		\begin{itemize}
			\item The Venn Diagram shows the number of elements in each subset of set $S$.
			\item If $P(A) = 3/10$ and $P(B) = 1/2$, find the values of $x$ and $y$
		\end{itemize}
		%----------------------------------------%
		%Venn Diagrams
	\end{frame}
	
	
	
	\begin{frame}
		\frametitle{Counting Sets with Venn Diagrams}
		\Large
		\vspace{-1cm}
		%----------------------------------------%
		\begin{itemize}
			\item The total number of items in the data set is $x+y+5$
			\item There are $x+1$ items in Area $A$
			\item There are $x+y$ items in Area $B$
			\item We can say
			\[ P(A) = \frac{3}{10} = \frac{x+1}{x+y+5}\]
			\[ P(B) = \frac{1}{2} = \frac{x+y}{x+y+5} \]
		\end{itemize}
		%----------------------------------------%
	\end{frame}
	
	
	
	\begin{frame}
		\frametitle{Counting Sets with Venn Diagrams}
		\Large
		\vspace{-2.5cm}
		\textbf{Cross Multiplication}
		\[ P(A) = \frac{3}{10} = \frac{x+1}{x+y+5}\]
		
	\end{frame}
	
	
	\begin{frame}
		\frametitle{Counting Sets with Venn Diagrams}
		\Large
		\vspace{-2.5cm}
		\textbf{Cross Multiplication}
		\[ P(B) = \frac{1}{2} = \frac{x+y}{x+y+5} \]
		
	\end{frame}
	
	
	\begin{frame}
		\frametitle{Counting Sets with Venn Diagrams}
		\Large
		\vspace{-2.8cm}
		\textbf{Simultaneous Equations}
		\begin{itemize}
			\item[1)] $7x-3y=5$
			\item[2)] $x+y=5$
		\end{itemize}
	\end{frame}
	
	\begin{frame}
		\frametitle{Counting Sets with Venn Diagrams}
		\Large
		\vspace{-2.8cm}
		\textbf{Simultaneous Equations}
		\begin{itemize}
			\item $7x-3y=5$
			\item $x+y=5$
		\end{itemize}
	\end{frame}
\end{document}