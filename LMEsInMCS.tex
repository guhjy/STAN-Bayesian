\newpage
\section{Using LME for method comparison}
Due to the prevalence of modern statistical software, \citet{BXC2008} advocates the adoption of computer based approaches, such as LME models, to method comparison studies. \citet{BXC2008} remarks upon `by-hand' approaches advocated in \citet{BA99} discouragingly, describing them as tedious, unnecessary and `outdated'. Rather than using the `by hand' methods, estimates for required LME parameters can be read directly from program output. Furthermore, using computer approaches removes constraints associated with `by-hand' approaches, such as the need for the design to be perfectly balanced.

\subsection{Roy's methodology}

For the purposes of comparing two methods of measurement, \citet{roy} presents a methodology utilizing linear mixed effects model. This methodology provides for the formal testing of inter-method bias, between-subject variability and within-subject variability of two methods. The formulation contains a Kronecker product covariance structure in a doubly multivariate setup. By doubly multivariate set up, Roy means that the information on each patient or item is multivariate at two levels, the number of methods and number of replicated measurements. Further to \citet{lam}, it is assumed that the replicates are linked over time. However it is easy to modify to the unlinked case.

\citet{roy} sets out three criteria for two methods to be considered in agreement. Firstly that there be no significant bias. Second that there is no difference in the between-subject variabilities, and lastly that there is no significant difference in the within-subject variabilities. Roy further proposes examination of the the overall variability by considering the second and third criteria be examined jointly. Should both the second and third criteria be fulfilled, then the overall variabilities of both methods would be equal.

A formal test for inter-method bias can be implemented by examining the fixed effects of the model. This is common to well known classical linear model methodologies. The null hypotheses, that both methods have the same mean, which is tested against the alternative hypothesis, that both methods have different means.
The inter-method bias and necessary $t-$value and $p-$value are presented in computer output. A decision on whether the first of Roy's criteria is fulfilled can be based on these values.

Importantly \citet{roy} further proposes a series of three tests on the variance components of an LME model, which allow decisions on the second and third of Roy's criteria. For these tests, four candidate LME models are constructed. The differences in the models are specifically in how the the $D$ and $\Lambda$ matrices are constructed, using either an unstructured form or a compound symmetry form. To illustrate these differences, consider a generic matrix $A$,

\[
\boldsymbol{A} = \left( \begin{array}{cc}
    a_{11} & a_{12}  \\
    a_{21} & a_{22}  \\
    \end{array}\right).
\]

A symmetric matrix allows the diagonal terms $a_{11}$ and $a_{22}$ to differ. The compound symmetry structure requires that both of these terms be equal, i.e $a_{11} = a_{22}$.

The first model acts as an alternative hypothesis to be compared against each of three other models, acting as null hypothesis models, successively. The models are compared using the likelihood ratio test. Likelihood ratio tests are a class of tests based on the comparison of the values of the likelihood functions of two candidate models. LRTs can be used to test hypotheses about covariance parameters or fixed effects parameters in the context of LMEs. The test statistic for the likelihood ratio test is the difference of the log-likelihood functions, multiplied by $-2$.
The probability distribution of the test statistic is approximated by the $\chi^2$ distribution with ($\nu_{1} - \nu_{2}$) degrees of freedom, where $\nu_{1}$ and $\nu_{2}$ are the degrees of freedom of models 1 and 2 respectively. Each of these three test shall be examined in more detail shortly.

\subsection{Correlation}
In addition to the variability tests, Roy advises that it is preferable that a correlation of greater than $0.82$ exist for two methods to be considered interchangeable. However if two methods fulfil all the other conditions for agreement, failure to comply with this one can be overlooked. Indeed Roy demonstrates that placing undue importance to it can lead to incorrect conclusions. \citet{roy} remarks that current computer implementations only gives overall correlation coefficients, but not their variances. Consequently it is not possible to carry out inferences based on all overall correlation coefficients.

%--------------------------------------------------%
\subsection{Variability test 1}
The first test determines whether or not both methods $A$ and $B$ have the same between-subject variability, further to the second of Roy's criteria.
\begin{eqnarray*}
H_{0}: \mbox{ }d_{A}  = d_{B} \\
H_{A}: \mbox{ }d_{A}  \neq d_{B}
\end{eqnarray*}
This test is facilitated by constructing a model specifying a symmetric form for $D$ (i.e. the alternative model) and comparing it with a model that has compound symmetric form for $D$ (i.e. the null model). For this test $\boldsymbol{\hat{\Lambda}}$ has a symmetric form for both models, and will be the same for both.

%---------------------------------------------%
\subsection{Variability test 2}

This test determines whether or not both methods $A$ and $B$ have the same within-subject variability, thus enabling a decision on the third of Roy's criteria.

\begin{eqnarray*}
H_{0}: \mbox{ }\lambda_{A}  = \lambda_{B} \\
H_{A}: \mbox{ }\lambda_{A}  = \lambda_{B}
\end{eqnarray*}

This model is performed in the same manner as the first test, only reversing the roles of $\boldsymbol{\hat{D}}$ and $\boldsymbol{\hat{\Lambda}}$. The null model is constructed a symmetric form for $\boldsymbol{\hat{\Lambda}}$ while the alternative model uses a compound symmetry form. This time $\boldsymbol{\hat{D}}$ has a symmetric form for both models, and will be the same for both.

As the within-subject variabilities are fundamental to the coefficient of repeatability, this variability test likelihood ratio test is equivalent to testing the equality of two coefficients of repeatability of two methods. In presenting the results of this test, \citet{roy} includes the coefficients of repeatability for both methods.

%-----------------------------------------------%
\subsection{Variability test 3}
The last of the variability test examines whether or not methods $A$ and $B$ have the same overall variability. This enables the joint consideration of second and third criteria.
\begin{eqnarray*}
H_{0}: \mbox{ }\sigma_{A}  = \sigma_{B} \\
H_{A}: \mbox{ }\sigma_{A}  = \sigma_{B}
\end{eqnarray*}

The null model is constructed a symmetric form for both $\boldsymbol{\hat{D}}$ and $\boldsymbol{\hat{\Lambda}}$ while the alternative model uses a compound symmetry form for both.

\subsection{Demonstration of Roy's testing}
Roy provides three case studies, using data sets well known in method comparison studies, to demonstrate how the methodology should be used. The first two examples used are from the `blood pressure' data set introduced by \citet{BA99}. The data set is a tabulation of simultaneous measurements of systolic blood pressure were made by each of two experienced observers (denoted `J' and `R') using a sphygmomanometer and by a semi-automatic blood pressure monitor (denoted `S'). Three sets of readings were made in quick succession. Roy compares the `J' and `S' methods in the first of her examples.

The inter-method bias between the two method is found to be $15.62$ , with a $t-$value of $-7.64$, with a $p-$value of less than $0.0001$. Consequently there is a significant inter-method bias present between methods $J$ and $S$, and the first of the Roy's three agreement criteria is unfulfilled.

Next, the first variability test is carried out, yielding maximum likelihood estimates of the between-subject variance covariance matrix, for both the null model, in compound symmetry (CS) form, and the alternative model in symmetric (symm) form. These matrices are determined to be as follows;
\[
\boldsymbol{\hat{D}}_{CS} = \left( \begin{array}{cc}
    946.50 & 784.32  \\
    784.32 & 946.50  \\
    \end{array}\right),
\hspace{1.5cm}
    \boldsymbol{\hat{D}}_{Symm} = \left( \begin{array}{cc}
    923.98 & 785.24  \\
    785.24 & 971.30  \\
    \end{array}\right).
\]

A likelihood ratio test is perform to compare both candidate models. The log-likelihood of the null model is $-2030.7$, and for the alternative model $-2030.8$. The test statistic, presented with greater precision than the log-likelihoods, is $0.1592$. The $p-$value is $0.6958$. Consequently we fail to reject the null model, and by extension, conclude that the hypothesis that methods $J$ and $S$ have the same between-subject variability. Thus the second of the criteria is fulfilled.

The second variability test determines maximum likelihood estimates of the within-subject variance covariance matrix, for both the null model, in CS form, and the alternative model in symmetric form.

\[
\boldsymbol{\hat{\Lambda}_{CS}} = \left( \begin{array}{cc}
    60.27  & 16.06  \\
    16.06  & 60.27  \\
    \end{array}\right),
\hspace{1.5cm}
\boldsymbol{\hat{\Lambda}}_{Symm} = \left( \begin{array}{cc}
    37.40 & 16.06  \\
    16.06 & 83.14  \\
    \end{array}\right).
\]

Again, A likelihood ratio test is perform to compare both candidate models. The log-likelihood of the alternative model model is $-2045.0$. As before, the null model has a log-likelihood of $-2030.7$. The test statistic is computed as $28.617$, again presented with greater precision. The $p-$value is less than $0.0001$. In this case we reject the null hypothesis of equal within-subject variability. Consequently the third of Roy's criteria is unfulfilled.
The coefficient of repeatability for methods $J$ and $S$ are found to be 16.95 mmHg and 25.28 mmHg respectively.

The last of the three variability tests is carried out to compare the overall variabilities of both methods.
With the null model the MLE of the within-subject variance covariance matrix is given below. The overall variabilities for the null and alternative models, respectively, are determined to be as follows;
\[
\boldsymbol{\hat{\Sigma}}_{CS} = \left( \begin{array}{cc}
    1007.92  & 801.65  \\
    801.65  & 1007.92  \\
    \end{array}\right),
    \hspace{1.5cm}
\boldsymbol{\hat{\Sigma}}_{Symm} = \left( \begin{array}{cc}
    961.38 & 801.40  \\
    801.40 & 1054.43  \\
    \end{array}\right),
\]

The log-likelihood of the alternative model model is $-2045.2$, and again, the null model has a log-likelihood of $-2030.7$. The test statistic is $28.884$, and the $p-$value is less than $0.0001$. The null hypothesis, that both methods have equal overall variability, is rejected. Further to the second variability test, it is known that this difference is specifically due to the difference of within-subject variabilities.

Lastly, Roy considers the overall correlation coefficient. The diagonal blocks $\boldsymbol{\hat{r}_{\Omega}}_{ii}$ of the correlation matrix indicate an overall coefficient of $0.7959$. This is less than the threshold of 0.82 that Roy recommends.

\[
\boldsymbol{\hat{r}_{\Omega}}_{ii} = \left( \begin{array}{cc}
    1  & 0.7959  \\
    0.7959  & 1  \\
    \end{array}\right)
\]

The off-diagonal blocks of the overall correlation matrix $\boldsymbol{\hat{r}_{\Omega}}_{ii'}$ present the correlation coefficients further to \citet{hamlett}.
\[
\boldsymbol{\hat{r}_{\Omega}}_{ii'} = \left( \begin{array}{cc}
    0.9611  & 0.7799  \\
    0.7799  & 0.9212  \\
    \end{array}\right).
\]

The overall conclusion of the procedure is that method $J$ and $S$ are not in agreement, specifically due to the within-subject variability, and the inter-method bias. The repeatability coefficients are substantially different, with the coefficient for method $S$ being 49\% larger than for method $J$. Additionally the overall correlation coefficient did not exceed the recommended threshold of $0.82$.
