\documentclass[12pt]{article}

\usepackage{amsmath}
\usepackage{amssymb}
\usepackage{framed}

\begin{document}


\Huge
\[\mbox{Statistics and Probability}\]
\LARGE
\[\mbox{Probability Functions}\]

\Large
\[\mbox{www.stats-lab.com}\]
\[\mbox{Twitter: @StatsLabDublin}\]

\newpage

%------------------------------------------------------------------------------------%
{\LARGE
\begin{center}
\textbf{Probability Functions}\\
\end{center}
\bigskip
Suppose $X$ is a random variable with probability density function
\[f_X(x) = a + bx^2\] over the range $(0,2)$, where $a$ and $b$ are constants.
The mean value of $X$ is 1.5.

\begin{enumerate}
\item Find $a$ and $b$,
\item Find $F_X(x)$, the cumulative distribution function of $X$,
\item Find the variance of $X$.
\end{enumerate}
}

\newpage
%------------------------------------------------------------------------------------
{\LARGE
\begin{center}
\textbf{Probability Functions}\\
\end{center}
\bigskip
\textbf{Question 1}\\
\begin{itemize}
\item Compute the coefficients $a$ and $b$.
\end{itemize}

\textbf{Remarks:}\\
Total area under the curve defined by the probability density function (between 0 and 2) must equal 1. (by definition.)\\ \bigskip

The expected value $\mu$ (or $E(X)$) of random variable $X$ is 1.5.


}

\newpage
%------------------------------------------------------------------------------------%
{\LARGE
\begin{center}
\textbf{Probability Functions}\\
\end{center}
\bigskip
\textbf{Remark 1}\\
Total area under the curve defined by the probability density function ( between 0 and 2) must equal 1. (by definition.)

\[\int^{2}_{0} a+bx^2\; dx = 1\]
}
\newpage
%------------------------------------------------------------------------------------%
{\LARGE
\begin{center}
\textbf{Probability Functions}\\
\end{center}
\bigskip
\textbf{Remark 2}\\
The expected value $\mu$ (or $E(X)$) of random variable $X$ is 1.5.

\[ \mu = \int^{2}_{0} x f_X(x)\; dx = 1.5\]
\[ \mu = \int^{2}_{0} x (a+bx^2)\; dx = 1.5\]

}
\newpage
{\LARGE
\begin{center}
\textbf{Probability Functions}\\
\end{center}
\bigskip
{ \Huge
\begin{itemize}
\item $2a + \frac{8b}{3} = 1$
\item $2a + 4b = 1.5$
\end{itemize}
}
}
\newpage
%------------------------------------------------------------------------------------
{\LARGE
\begin{center}
\textbf{Probability Functions}\\
\end{center}
\bigskip
\textbf{Question 2}\\
\begin{itemize}
\item Find $F_X(x)$, the cumulative distribution function of $X$,
\end{itemize}
\[F_X(x) = \int_{0}^x f_X(x)\,dx.\]
\[F_X(x) = \int_{0}^x (a+bx^2)\,dx.\]
}

\newpage
{\LARGE
\begin{center}
\textbf{Probability Functions}\\
\end{center}
\[F_X(x) = \int_{0}^x \frac{3}{8}x^2\,dx.\]
}
%------------------------------------------------------------------------------------%
\newpage
{\LARGE
\begin{center}
\textbf{Probability Functions}\\
\end{center}
\bigskip
\textbf{Computing the Variance}\\ \bigskip
Definitions:
\[ \operatorname{Var}(X) =\sigma^2 =\int (x-\mu)^2 \, f_X(x) \, dx\, \]
\[ \operatorname{Var}(X) =\sigma^2  =\int x^2 \, f_X(x) \, dx\, - \mu^2\]
\bigskip

For this example :
\[ = \int^{2}_{0} x^2 \, \left( a+bx^2 \right) \, dx\, -(1.5)^2\]

\[   =\int^{2}_{0} (ax^2 + bx^4) \, dx\, - (1.5)^2\]
}

%------------------------------------------------------------------------------------%
\newpage

\end{document}