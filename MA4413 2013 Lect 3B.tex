\documentclass[a4]{beamer}
\usepackage{amssymb}
\usepackage{graphicx}
\usepackage{subfigure}
\usepackage{newlfont}
\usepackage{amsmath,amsthm,amsfonts}
%\usepackage{beamerthemesplit}
\usepackage{pgf,pgfarrows,pgfnodes,pgfautomata,pgfheaps,pgfshade}
\usepackage{mathptmx}  % Font Family
\usepackage{helvet}   % Font Family
\usepackage{color}
\mode<presentation> {
 \usetheme{Default} % was
 \useinnertheme{rounded}
 \useoutertheme{infolines}
 \usefonttheme{serif}
 %\usecolortheme{wolverine}
% \usecolortheme{rose}
\usefonttheme{structurebold}
}
\setbeamercovered{dynamic}

\title[MA4413]{Statistics for Computing \\ {\normalsize MA4413 Lecture 3A}}
\author[Kevin O'Brien]{Kevin O'Brien \\ {\scriptsize Kevin.obrien@ul.ie}}
\date{Autumn Semester 2013}
\institute[Maths \& Stats]{Dept. of Mathematics \& Statistics, \\ University \textit{of} Limerick}

\renewcommand{\arraystretch}{1.5}

\begin{document}

\begin{frame}
\titlepage
\end{frame}
\frame{
\begin{itemize}
\item Binomial Coefficients / The Choose Operator
\item Definition: The Probability Mass Functions (pmf)
\item Binomial Distribution : Example
\end{itemize}
}
\frame{
\frametitle{Binomial Coefficients}

In the last class, we came across binomial coefficients. Informally, binomial coefficients are the number of ways $k$ items can be selected from a group of $n$ items. 
The binomial coefficient indexed by n and k is usually written as $^nC_k$ or
\[ {n \choose k}\].
$C$ is colloqially known as the ``choose operator".

\[ {n \choose k} = \frac{n!}{k! \times (n-k)!} \]

(We call the operator the choose operator. We will use both notations interchangeably.)
 

}
\frame{
\frametitle{Binomial Coefficients}

\begin{itemize}
\item $n!$ and $k!$ are the coefficients of $n$ and $k$ respectively.
\item $n! = n \times (n-1) \times (n-2) \times \ldots \times 2 \times 1$
\item For example $5! = 5\times4\times3\times2\times1 = 120$
\item $n! = n \times (n-1)!$
\item Importantly $0! = 1$ not 0.
\end{itemize}
\[ {6 \choose 2} = \frac{6!}{2! \times (6-2)!} = \frac{6!}{2! \times 4!}  \]\[\mbox{   } = \frac{6 \times 5 \times 4!}{2! \times 4!} 
 = 30/2 =15 \]
More examples of Binomial coefficients on blackboard.
}
%---------------------------------------------------------------------------%
\frame{
\frametitle{Probability Mass Function}
(Formally defining something mentioned previously)
\begin{itemize} \item a probability mass function (pmf) is a \textbf{\emph{function}}
that gives the probability that a discrete random variable is exactly equal to some
value.
\[P(X=k)\]
\item The probability mass function is often the primary means of defining a discrete
probability distribution
\item It is conventional to present the probability mass function in the form of a table.
\item The p.m.f of a value $k$ is often denoted $f(k)$.
\end{itemize}
}
%--------------------------------------------------------------------------------------%
\frame{
\frametitle{Probability Tables}
In the \textbf{Sulis} workspace there are two important tables used for this part of the course.


This class will feature a demonstration on how to read those tables.
\begin{itemize}
\item The Cumulative Binomial Tables (Murdoch Barnes Tables 1)
\item The Cumulative Poisson Tables (Murdoch Barnes Tables 2)
\end{itemize}

Please get a copy of each as soon as possible.

}

%---------------------------------------------------------------------------%
\frame{
\frametitle{Probability Tables}
\begin{itemize}
\item For some value $r$ the tables record the probability of $P(X \geq r)$.
\item The Student is required to locate the appropriate column based on the parameter values for the distribution in question.
\item A copy of the Murdoch Barnes Tables will be furnished to the student in the End of Year Exam. The Tables are not required for the first mid-term exam.
\item Knowledge of the sample space, partitioning of the sample points, and the complement rule are advised.
\end{itemize}
}
%---------------------------------------------------------------------------%


\frame{
\frametitle{Binomial Distribution : Using Tables}
It is estimated by a particular bank that 25\% of credit card customers pay only the minimum amount due on their monthly credit card bill and do not pay the total amount due. 50 credit card customers are randomly selected.
\begin{enumerate}
\item (3 marks)	What is the probability that 9 or more of the selected customers pay only the minimum amount due?
\item (3 marks) What is the probability that less than 6 of the selected customers pay only the minimum amount due?
\item (3 marks)	What is the probability that more than 5 but less than 10 of the selected customers pay only the minimum amount due?
\end{enumerate}

}

\frame{
\frametitle{Binomial Distribution : Using Tables}
Demonstration on Blackboard re: how to use tables in class.
\begin{enumerate}
\item $P(X \geq 9) = 0.9084$
\item $P(X < 6) = 1- P(X \geq 6) =1 - 0.9930 = 0.0070$
\item $P(6 \leq X \leq 9) = P(X \geq 6) - P(X \geq 10) = 0.9930 - 0.8363 = 0.1567$
\end{enumerate}

}


%------------------------------------------------------------------%
\frame{
\frametitle{Binomial Distribution: Expected Value and Variance}


If the random variable X has a binomial distribution with parameters n
and p, we write
\[ X \sim B(n,p) \]

Expectation and Variance
If $X \sim B(n,p)$, then:

\begin{itemize}
\item Expected Value of X : $E(X) = np$
\item Variance of X : $Var(X) = np(1-p)$
\end{itemize}

Suppose n=3 and p=0.5 
Then $E(X) = 1.5$ and $V(X) = 0.75$.

Remark: Referring to the expected value and variance may be used to validate
the assumption of a binomial distribution.

}
%---------------------------------------------------------------------------%
\frame{
\frametitle{The Geometric Distribution}
\begin{itemize}
\item The Geometric distribution is related to the Binomial distribution in that
both are based on independent trials in which the probability of success
is constant and equal to p.
\item However, a Geometric random variable is the number of trials until the
first failure, whereas a Binomial random variable is the number of
successes in n trials.
\item The Geometric distributions is often used in IT security applications.
\end{itemize}
}
%---------------------------------------------------------------------------%
\frame{
\frametitle{The Geometric Distribution}

Suppose that a random experiment has two possible outcomes, success
with probability p and failure with probability 1-p .


The experiment is repeated until a success happens. The number of
trials before the success is a random variable X computed as follows

\[P(X = k) = (1-p)^{(k-1)}\times p \]


(i.e. The probability that first success is on the k-th trial)
}


%---------------------------------------------------------------------------%
\frame{
\frametitle{The Geometric Distribution: Notation}

If X has a geometric distribution with parameter p, we write
\[X \sim Geo(p) \]
Expectation and Variance
If $X \sim Geo(p)$, then:

\begin{itemize}
\item Expected Value of X : E(X) = 1/p
\item Variance of X : Var(X) = $(1-p)/p^2$.
\end{itemize}
}

%---------------------------------------------------------------------------%
\frame{
\frametitle{Poisson Experiment}
A Poisson experiment is a statistical experiment that has the following properties:
\begin{itemize}
\item The experiment results in outcomes that can be classified as successes or failures.
\item The average number of successes (m) that occurs in a specified region is known.
\item The probability that a success will occur is proportional to the size of the region.
\item The probability that a success will occur in an extremely small region is virtually zero.
\end{itemize}
Note that the specified region could take many forms. For instance, it could be a length, an area, a volume, a period of time, etc.
}

%---------------------------------------------------------------------------%
\frame{
\frametitle{Poisson Probability Distribution}
\begin{itemize}
\item
A Poisson random variable is the number of successes that result from a Poisson experiment.

\item 
The probability distribution of a Poisson random variable is called a Poisson distribution.

\end{itemize}
}

%---------------------------------------------------------------------------%
\frame{
\frametitle{The Poisson Probability Distribution}
\begin{itemize}
\item The number of occurrences in a unit period (or space)
\item The expected number of occurrences is $m$
\item Given the mean number of successes ($m$) that occur in a specified region, we can compute the Poisson probability based on the following formula (next slide).
\end{itemize}
}

%---------------------------------------------------------------------------%
\frame{
\frametitle{Poisson Formulae}
The probability that there will be $k$ occurrences in a unit time period is denoted $P(X=k)$, and is computed as follows.
\Large
\[ P(X = k)=\frac{m^k e^{-m}}{k!} \]

}
%---------------------------------------------------------------------------%
\frame{
\frametitle{Poisson Formulae}
Given that there is on average 2 occurrences per hour, what is the probability of no occurrences in the next hour? \\ i.e. Compute $P(X=0)$ given that $m=2$
\Large
\[ P(X = 0)=\frac{2^0 e^{-2}}{0!} \]
\normalsize
\begin{itemize}
\item $2^0$ = 1
\item $0!$ = 1
\end{itemize}
The equation reduces to
\[ P(X = 0)=e^{-2} = 0.1353\]
}
%---------------------------------------------------------------------------%
\frame{
\frametitle{Poisson Formulae}
What is the probability of one occurrences in the next hour? \\ i.e. Compute $P(X=1)$ given that $m=2$
\Large
\[ P(X = 1)=\frac{2^1 e^{-2}}{1!} \]
\normalsize
\begin{itemize}
\item $2^1$ = 2
\item $1!$ = 1
\end{itemize}
The equation reduces to
\[ P(X = 1) = 2 \times e^{-2} = 0.2706\]
}
%---------------------------------------------------------------------------%



%---------------------------------------------------------------------%
\begin{frame}
\frametitle{The Cumulative Distribution Function}
\begin{itemize}
\item The Cumulative Distribution Function, denoted $F(x)$, is a common way that the probabilities
of a random variable (both discrete and continuous) can be summarized.
\item The Cumulative Distribution Function, which also can be
described by a formula or summarized in a table, is defined as:
\[F(x) = P(X \leq x) \]
\item The notation for a cumulative distribution function, F(x), entails using a capital
"F".  (The notation for a probability mass or density function, f(x), i.e. using a lowercase "f". The notation is not interchangeable.
\end{itemize}
\end{frame}

%---------------------------------------------------------------------%
\begin{frame}
\frametitle{Useful Results}
(Demonstration on the blackboard re: partitioning of the sample space, using examples on next slide)
\begin{itemize}
\item $P(X \leq 1) = P(X=0) + P(X=1)$
\item $P(X \leq r) = P(X=0)+ P(X=1) + \ldots P(X= r)$
\item $P(X \leq 0) = P(X=0)$
\item $P(X = r) = P(X \geq r ) - P(X \geq r + 1)$
\item \textbf{Complement Rule}: $P(X \leq r-1) = P(X < r) = 1 - P(X \geq r)$
\item \textbf{Interval Rule}:$ P(a \leq X \leq  b)= P(X \geq a) - P(X \geq b + 1).$
\end{itemize}
For the binomial distribution, if the probability of success is greater than 0.5, instead of
considering the number of successes, to use the table we consider
the number of failures.
\end{frame}
%---------------------------------------------------------------------%


%---------------------------------------------------------------------%
\begin{frame}
\frametitle{Binomial Example 1}
Suppose a signal of 100 bits is transmitted and the probability of
sending a bit correctly is 0.9. What is the probability of
\begin{enumerate}
\item at least 10 errors
\item exactly 7 errors
\item Between 5 and 15 errors (inclusively).
\end{enumerate}
\end{frame}
%---------------------------------------------------------------------%
\begin{frame}
\frametitle{Binomial Example 1}
\begin{itemize}
\item Since the probability of success is 0.9. We consider the distribution
of the number of failures (errors).
\item We reverse the definition of `success' and `failure'. Success is now defined as an error.
\item The probability that a bit is sent incorrectly is 0.1.
\item Let X be the total number of errors. $X \sim B(100, 0.1)$.
\item Answer : $P(X \geq 10) = 0.5487$.
\item $P(X = 7)=P(X \geq 7) - P(X \geq 8) =0.8828 - 0.7939 = 0.0889$.
\item $P(5 \leq X  \leq 15) = P(X \geq 5) - P(X \geq 16) =0.9763 - 0.0399 = 0.9364$
\end{itemize}
\end{frame}

%---------------------------------------------------------------------------%
\frame{
\frametitle{The Poisson Probability Distribution}
\begin{itemize}
\item A Poisson random variable is the number of successes that result from a Poisson experiment.
\item The probability distribution of a Poisson random variable is called a Poisson distribution.
\item Very Important: This distribution describes the number of occurrences in a \textbf{\emph{unit period (or space)}}
\item Very Important: The expected number of occurrences is $m$
\end{itemize}
}
\begin{frame}
\frametitle{The Poisson Probability Distribution}
We use the following notation.
\[X \sim Poisson(m) \]
Note the expected number of occurrences per unit time is conventionally denoted $\lambda$ rather than $m$.
\bigskip
As the Murdoch Barnes cumulative Poisson Tables (Table 2) use $m$, so shall we. Recall that Tables 2 gives values of the probability $P(X \geq r )$, when X has a Poisson distribution with
parameter $m$.

\end{frame}
%---------------------------------------------------------------------%
%---------------------------------------------------------------------------%
\begin{frame}
\frametitle{The Poisson Probability Distribution}
Consider cars passing a point on a rarely used country road. Is this a Poisson Random Variable?
Suppose
\begin{enumerate}
\item Arrivals occur at an average rate of $m$ cars per unit time.
\item The probability of an arrival in an interval of length k
is constant.
\item The number of arrivals in two non-overlapping
intervals of time are independent.
\end{enumerate}
This would be an appropriate use of the Poisson Distribution.
\end{frame}

%---------------------------------------------------------------------%
\begin{frame}
\frametitle{Changing the unit time.}

\begin{itemize}
\item The number of arrivals, X, in an interval of length $t$ has a
Poisson distribution with parameter $\mu = mt$.
\item $m$ is the expected number of arrivals in a unit time period.
\item $\mu$ is the expected number of arrivals in a time period $t$, that is different from the unit time period.
\item Put simply : if we change the time period in question, we adjust the Poisson mean accordingly.
\item If 10 occurrences are expected in 1 hour, then 5 are expected in 30 minutes. Likewise, 20 occurrences are expected in 2 hours, and so on.
\item (Remark : we will not use $\mu$ in this context anymore).
\end{itemize}
\end{frame}


%---------------------------------------------------------------------%
\begin{frame}
\frametitle{Poisson Example}
A motor dealership which specializes in agricultural machinery sells one vehicle every 2 days, on average. Answer the following questions.
    \begin{enumerate}
    \item  What is the probability that the dealership sells at least one vehicle in one particular day?
    \item  What is the probability that the dealership will sell exactly one vehicle in one particular day?
    \item  What is the probability that the dealership will sell 4 vehicles or more in a six day working week?
    \end{enumerate}
\end{frame}

%---------------------------------------------------------------------%
\begin{frame}
\frametitle{Poisson Example}

    \begin{enumerate}
    \item Expected Occurrences per Day: m = 0.5
    \item Probability that the dealership sells at least one vehicle in one particular day?
    \[ P(X \geq 1) = 0.3935 \]
    \item Probability that the dealership will sell exactly one vehicle in one particular day?
    \[ P(X = 1) = P(X \geq 1) - P(X \geq 2)  = 0.3935 - 0.0902 = 0.3031\]
    \item Probability that the dealership will sell 4 vehicles or more in a six day working week?
    \begin{itemize}
    \item For a 6 day week, m=3
    \item $P(X \geq 4) = 0.3528$
    \end{itemize}
    \end{enumerate}
\end{frame}

%---------------------------------------------------------------------%
\begin{frame}
\frametitle{Knowing which distribution to use}
\begin{itemize}
\item For the end of semester examination, you will be required to know when it is appropriate to use the Poisson distribution, and when to use the binomial distribution.
\item Recall the key parameters of each distribution.
\item Binomial : number of \textbf{\emph{successes}} in $n$ \textbf{\emph{independent trials}}.
\item Poisson : number of \textbf{\emph{occurrences}} in a \textbf{\emph{unit space}}.
\end{itemize}
\end{frame}

\end{document}
%---------------------------------------------------------------------%
\begin{frame}
\frametitle{Poisson Approximation of the Binomial}

\begin{itemize}
\item The Poisson distribution can sometimes be used to approximate the
binomial distribution
\item When the number of observations n is large, and the success probability
p is small, the $Bin(n,p)$ distribution approaches the Poisson distribution
with the parameter given by $m = np$.
\item This is useful since the computations involved in calculating binomial
probabilities are greatly reduced.
\item As a rule of thumb, n should be greater than 50 with p very small, such
that np should be less than 5.
\item If the value of p is very high, the definition of what constitutes a
``success" or ``failure" can be switched.
\end{itemize}
\end{frame}

%---------------------------------------------------------------------%
\begin{frame}
\frametitle{Poisson Approximation: Example}

\begin{itemize}
\item Suppose we sample 1000 items from a production line that is producing, on
average, $0.1\%$ defective components.
\item Using the binomial distribution, the probability of exactly 3 defective items in
our sample is
\[P(X = 3) = ^{1000}C_{3} \times 0.001^{3} \times 0.999^{997}\]
\end{itemize}
\end{frame}

%---------------------------------------------------------------------%
\begin{frame}
\frametitle{Poisson Approximation: Example}
Lets compute each of the component terms individually.

\begin{itemize}
\item $^{1000}C_{3}$
\[^{1000}C_{3} = \frac{1000 \times 999 \times 998}{3 \times 2 \times 1} = 166,167,000\]
\item $0.001^3$
\[0.001^3 = 0.000000001\]
\item $0.999^{997}$
\[0.999^{997} = 0.36880\]
\end{itemize}


Multiply these three values to compute the binomial probability
$P(X = 3) = 0.06128$
\end{frame}

\begin{frame}
\frametitle{Poisson Approximation: Example}
\begin{itemize}
\item Lets use the Poisson distribution to approximate a solution.
\item First check that $n \geq 50$ and $np < 5$ (Yes to both).
\item We choose as our parameter value $m = np = 1000 \times 0.001 = 1$
\end{itemize}
\[P(X = 3) = \frac{e^{-1} \times 1^3}{3!} = \frac{e^{-1}}{6} = \frac{0.36787}{6} = 0.06131 \]
Compare this answer with the Binomial probability
P(X = 3) = 0.06128.
Very good approximation, with much less computation effort.
\end{frame}

\end{document}
