
%------------------------------------------------------------%
\frame{
\frametitle{Random Variables}
\begin{itemize} \item The outcome of an experiment need not be a number, for example, the outcome when a coin is tossed can be `heads' or `tails'. \item
However, we often want to represent outcomes as numbers. \item
A \textbf{\emph{random variable}} is a function that associates a unique numerical value with every outcome of an experiment.
\item The value of the random variable will vary from trial to trial as the experiment is repeated.
\item Numeric values can be assigned to outcomes that are not usually considered numeric. \item For example, we could assign a `head' a value of $0$, and a `tail' a value of $1$, or vice versa.
\end{itemize}
}
%------------------------------------------------------------%
\frame{
\frametitle{Random Variables}
There are two types of random variable - discrete and continuous. The distinction between both types will be important later on in the course.\\ \bigskip

\textbf{Examples}
\begin{itemize}
\item A coin is tossed ten times. The random variable X is the number of tails that are noted.
X can only take the values $\{0, 1, ..., 10\}$, so $X$ is a discrete random variable.
\item A light bulb is burned until it burns out. The random variable Y is its lifetime in hours.
Y can take any positive real value, so Y is a continuous random variable.
\end{itemize}
}

%--------------------------------------------------------------------------------%
\frame{
\frametitle{Discrete Random Variable}
\begin{itemize}
\item A discrete random variable is one which may take on only a countable number of distinct values such as $\{0, 1, 2, 3, 4, ... \}$.\item Discrete random variables are usually (but not necessarily) counts. \item If a random variable can take only a finite number of distinct values, then it must be discrete. \item Examples of discrete random variables include the number of children in a family, the Friday night attendance at a cinema, the number of patients in a doctor's surgery, the number of defective light bulbs in a box of ten.
    \end{itemize}
}


%--------------------------------------------------------------------------------%
\frame{
\frametitle{Continuous Random Variable}
\begin{itemize} \item
A continuous random variable is one which takes an infinite number of possible values. \item Continuous random variables are usually measurements. \item Examples include height, weight, the amount of sugar in an orange, the time required to run a computer simulation. \end{itemize}

}
\end{document}