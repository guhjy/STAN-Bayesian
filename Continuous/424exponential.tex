\documentclass{beamer}

\usepackage{amsmath}
\usepackage{amssymb}


\begin{document}

\begin{frame}
\Huge
\[\mbox{Introduction to Probability }\]
\huge
\[\mbox{Exponential Distribution}\]

\Large
\[\mbox{www.Stats-Lab.com}\]
\[\mbox{Twitter: @StatsLabDublin}\]
\end{frame}


\begin{frame}[fragile]
\frametitle{Exponential Distribution}
\Large
The Exponential Distribution may be used to answer the following questions:
\begin{itemize}
\item How much time will elapse before an earthquake occurs in a given region?
\item How long do we need to wait before a customer enters a shop?
\item How long will it take before a call center receives the next phone call?
\item How long will a piece of machinery work without breaking down?
\end{itemize}
\end{frame}

\begin{frame}[fragile]
\frametitle{Exponential Distribution}
\Large
\begin{itemize}
\item All these questions concern the time we need to wait before a given event occurs. 
\item If this waiting time is unknown, it is often appropriate to think of it as a random variable having an \textbf{exponential distribution}.
\item The time $X$ we need to wait before an event occurs has an exponential distribution if the probability that the event occurs during a certain time interval is proportional to the length of that time interval.

\end{itemize}
\end{frame}

%------------------------------------------------------------------------%
\begin{frame}[fragile]
\frametitle{Probability density function}
\Large
The probability density function (PDF) of an exponential distribution is

\[
f(x;\lambda) = \begin{cases}
\lambda e^{-\lambda x}, & x \ge 0, \\
0, & x < 0.
\end{cases}\]
The parameter $\lambda$  is called \textbf{\emph{rate}} parameter. It is the inverse of the expected duration ($\mu$).\\ \bigskip

(If the expected duration is 5 (e.g. five minutes) then the rate parameter value is 0.2.)
\end{frame}

%------------------------------------------------------------------------%
\begin{frame}[fragile]
\frametitle{Exponential Distribution: Cumulative density function}
\Large
\vspace{-1cm}
The cumulative distribution function (CDF) of an exponential distribution is

\[
F(x;\lambda) = \begin{cases}
1-e^{-\lambda x}, & x \ge 0, \\
0, & x < 0.
\end{cases}\]

The CDF can be written as the probability of the lifetime being less than some value $x$.

\[ P(X \leq x) = 1-e^{-\lambda x} \]

\end{frame}

%------------------------------------------------------------------------%
\begin{frame}[fragile]
\frametitle{Exponential Distribution: Expected Value and Variance}
\Large
\vspace{-1.2cm}
The expected value of an exponential random variable $X$ is:

\[
E[X] = \frac{1}{\lambda}\]
The variance of an exponential random variable $X$ is:

\[
V[X] = \frac{1}{\lambda^2}\]

\end{frame}

%------------------------------------------------------------------------%
\begin{frame}[fragile]
\frametitle{Exponential Distribution: Example}
\Large
\vspace{-1.2cm}
Assume that the length of a phone call in minutes is an exponential random variable $X$ with parameter
$\lambda = 1/10$. \\ 
\vspace{0.5cm}
If someone arrives at a phone booth just before you arrive, find the probability that you
will have to wait \begin{itemize}
\item[(a)] less than 5 minutes,
\item[(b)] greater than 5 minutes,
\item[(c)] between 5 and 10 minutes.
\end{itemize}
Also compute the expected value and variance.
\end{frame}



%------------------------------------------------------------------------%
\begin{frame}[fragile]
\frametitle{Exponential Distribution: Example}
\vspace{-2.5cm}
\Large
\textbf{Part a}\\
Compute $P(X \leq 5)$ with $\lambda = 1/10$

\[ P(X \leq x) = 1-e^{-\lambda x} \]
\end{frame}

%------------------------------------------------------------------------%
\begin{frame}[fragile]
\frametitle{Exponential Distribution: Example}
\Large
\vspace{-1cm}
\textbf{Part a}\\
Compute $P(X \leq 5)$ with $\lambda = 1/10$

\begin{itemize}
\item $ P(X \leq x) = 1-e^{-\lambda x} $
\item $ P(X \leq 5) = 1-e^{-5/10}  $
\item $ P(X \leq 5) = 1-e^{-0.5} $
\item $ P(X \leq 5) = 1-0.6065 $
\item $ P(X \leq 5) = 0.3934 $
\end{itemize} 

\end{frame}




\begin{frame}[fragile]
\frametitle{Exponential Distribution: Example}
\vspace{-2.5cm}
\Large
\textbf{Part b}\\
Compute $P(X \geq 10)$ with $\lambda = 1/10$

\[ P(X \leq x) = 1-e^{-\lambda x} \]
Complement rule
\[ P(X \geq x) = 1- P(X \leq x) =  e^{-\lambda x} \]


\end{frame}

\begin{frame}[fragile]
\frametitle{Exponential Distribution: Example}
\vspace{-1.5cm}
\Large
\textbf{Part b}\\
Compute $P(X \geq 10)$ with $\lambda = 1/10$

\[ P(X \geq x) = e^{-\lambda x} \]

\begin{itemize}
\item $ P(X \geq x) = e^{-\lambda x} $
\item $ P(X \geq 10) = e^{-10/10}  $
\item $ P(X \geq 10) = e^{-1} $
\item $ P(X \geq 10) =  0.3678 $
\end{itemize} 
\end{frame}

%------------------------------------------------------------------------%
\begin{frame}[fragile]
\frametitle{Exponential Distribution: Example}
\vspace{-1.0cm}
\Large
\textbf{Part c}\\

Compute $P(5 \leq X \leq 10)$ with $\lambda = 1/10$  \\
\vspace{0.3cm}
\begin{itemize}
\item Probability of being inside this interval is the complement of being outside the interval.
\item The probability of being outside the interval is the composite event of being too low for the interval (i.e. $P( X \leq 5)$)
and being too high for the interval (i.e. $P( X \leq 10)$).
\end{itemize}
\[ P(5 \leq X \leq 10) = 1 - \left[ P( X \leq 5) + P( X \geq 10)  \right] \]
\end{frame}

\begin{frame}[fragile]
\frametitle{Exponential Distribution: Example}
\vspace{-1.0cm}
\Large
\textbf{Part c}\\
Compute $P(5 \leq X \leq 10)$ with $\lambda = 1/10$ 
\vspace{0.1cm}
\[ P(5 \leq X \leq 10) = 1 - \left[ P( X \leq 5) + P( X \geq 10)  \right] \]
\vspace{-0.6cm}
\begin{itemize}
\item \textbf{Too Low} $P(X \leq 5)$ = 0.3934
\vspace{0.2cm}
\item \textbf{Too High} $P(X \geq 10)$ = 0.3678
\vspace{0.2cm}
\item \textbf{Outside} $P( X \leq 5) + P( X \geq 10)$ = 0.7612
\vspace{0.2cm}
\item \textbf{Inside} $P(5 \leq X \leq 10)$ = 1 - 0.7612 = 0.2388
\end{itemize}
\end{frame}

\begin{frame}[fragile]
\frametitle{Exponential Distribution}
\Large
\vspace{-1.0cm}
\textbf{Expected Value and Variance}\\
The expected value of an exponential random variable $X$ is:

\[
E[X] = \frac{1}{\lambda} = \frac{1}{1/10} = 10 \]
The variance of an exponential random variable $X$ is:

\[
V[X] = \frac{1}{\lambda^2} = 100\]

\end{frame}



\begin{frame}[fragile]
\frametitle{Exponential Distribution}

\end{frame}



%------------------------------------------------------------------------%
\begin{frame}[fragile]
\frametitle{Exponential Distribution: Relationship to Poisson Mean}
\begin{itemize}
\item The Exponential Rate parameter ($\lambda$) is related to the Poisson mean (m)
\item If we expect 12 occurrences per hour - what is the rate of occurrences?
\item We would expected to wait 1/12 of an hour (i.e. 5 minutes) between occurrences.
\item Be mindful to keep your time units consistent, if working with both Poisson and Exponential.
\item If working in minutes, our rate parameter values is $\lambda$ = 0.20 (i.e. 1/5).
\item (This could be the basis of an exam question).
\end{itemize}
\end{frame}

\end{document}
