
\documentclass[12pt, a4paper]{article}
\usepackage{natbib}
\usepackage{vmargin}
\usepackage{graphicx}
\usepackage{epsfig}
\usepackage{subfigure}
%\usepackage{amscd}
\usepackage{amssymb}
\usepackage{subfigure}
\usepackage{amsbsy}
\usepackage{amsthm, amsmath}
%\usepackage[dvips]{graphicx}
\bibliographystyle{chicago}
\renewcommand{\baselinestretch}{1.1}

% left top textwidth textheight headheight % headsep footheight footskip
\setmargins{3.0cm}{2.5cm}{15.5 cm}{23.5cm}{0.25cm}{0cm}{0.5cm}{0.5cm}

\pagenumbering{arabic}


\begin{document}
\section{Work List}
\begin{enumerate}
\item ML v REML
\item Nested Models and LRTs
\item Generalized Lease Squares
\item Diagnostics
\item Simplifying GLS
\item Paper progression
\end{enumerate}

\section{Linear mixed effects models}

% http://www.artifex.org/~meiercl/R_statistics_guide.pdf
These models are used when there are both fixed and random effects that need to be incorporated into a model.

Fixed effects usually correspond to experimental treatments for which one has data for the entire population of samples corresponding to that treatment.

Random effects,on the other hand, are assigned in the case where we have measurements on a group of samples, and those
samples are taken from some larger sample pool, and are presumed to be representative.

As such, linear mixed effects models treat the error for fixed effects differently than the error for random effects.








\newpage
%--------------------------------------------------------------------Diagnostics%
\section{Diagnostics}

%http://www.artifex.org/~meiercl/R_statistics_guide.pdf
\subsection{Identifying outliers with a LME model object}

The process is slightly different than with standard LME model objects, since the \textbf{\emph{influence}}
function does not work on lme model objects. Given \textbf{\emph{mod.lme}}, we can use the plot function to
identify outliers.
%----------------------%
\subsection{Diagnostics for Random Effects}
Empirical best linear unbiased predictors EBLUPS provide the a useful way of diagnosing random effects.

EBLUPs are also known as ``shrinkage estimators" because they tend to be smaller than the estimated effects would be if they were computed by treating a random factor as if it was fixed (West etal )


\bibliography{DB-txfrbib}
\end{document}
