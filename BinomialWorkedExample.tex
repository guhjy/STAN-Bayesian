
4.B.1 The Binomial Probability Distribution

 

A binomial experiment is one that possesses the following properties:

1.   The experiment consists of n repeated trials;

 

2.   Each trial results in an outcome that may be classified as a success or a failure (hence the name, binomial);

 

3.   The probability of a success, denoted by p, remains constant from trial to trial and repeated trials are independent.

\end{frame}
%=======================================================================================================%
\begin{frame}
\frametitle{Binomial Probability Distribution}
\Large

The number of successes X in n trials of a binomial experiment is called a binomial random variable.

The probability distribution of the random variable X is called a binomial distribution, and is given by the formula:


P(X) = Cnx pxqn−x

where

n = the number of trials

x = 0, 1, 2, ... n

p = the probability of success in a single trial

q = the probability of failure in a single trial

(i.e. q = 1 − p)

Cnx is a combination value, found using the Choose operator.

 \end{frame}
%=======================================================================================================%
\begin{frame}
\frametitle{Binomial Probability Distribution}
\Large

P(X) gives the probability of successes in n binomial trials.

If p is the probability of success and q is the probability of failure in a binomial trial, then the expected number of successes in n trials (i.e. the mean value of the binomial distribution) is

E(X) = μ = np

\end{frame}
%=======================================================================================================%
\begin{frame}
\frametitle{Binomial Probability Distribution}
\Large
The variance of the binomial distribution is

V(X) = σ2 = npq

Note: In a binomial distribution, only 2 parameters, namely n and p, are needed to determine the probability.
 
 
 

MathsCast 6 : The binomial distribution

 The Binomial Dsitributuion is characheterized by the following parameters

 

1) N - the number of trials

2) P - The probability of a "success"

 

The word "success" means that the outcome is the outcome of interest.

If the outcome of interest is something like a flat tire, using the word "success" is coutner intuituive.


A binomial experiment (also known as a Bernoulli trial) is a statistical experiment that has the following properties:

The experiment consists of n repeated trials.


Each trial can result in just two possible outcomes. We call one of these outcomes a success and the other, a failure.

The probability of success, denoted by P, is the same on every trial.

The trials are independent; that is, the outcome on one trial does not affect the outcome on other trials.

MathsCast 6 : The binomial distribution

 The Binomial Dsitributuion is characheterized by the following parameters

 

1) N - the number of trials

2) P - The probability of a "success"

 

The word "success" means that the outcome is the outcome of interest.

If the outcome of interest is something like a flat tire, using the word "success" is coutner intuituive.


A binomial experiment (also known as a Bernoulli trial) is a statistical experiment that has the following properties:

The experiment consists of n repeated trials.


Each trial can result in just two possible outcomes. We call one of these outcomes a success and the other, a failure.

The probability of success, denoted by P, is the same on every trial.

The trials are independent; that is, the outcome on one trial does not affect the outcome on other trials.



