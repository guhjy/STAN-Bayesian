\documentclass[a4]{beamer}
\usepackage{amssymb}
\usepackage{graphicx}
\usepackage{multicol}
\usepackage{subfigure}
\usepackage{newlfont}
\usepackage{amsmath,amsthm,amsfonts}
%\usepackage{beamerthemesplit}
\usepackage{pgf,pgfarrows,pgfnodes,pgfautomata,pgfheaps,pgfshade}
\usepackage{mathptmx}  % Font Family
\usepackage{helvet}   % Font Family
\usepackage{color}
\usepackage{subfiles}

\mode<presentation> {
 \usetheme{Default} % was Frankfurt
 \useinnertheme{rounded}
 \useoutertheme{infolines}

 \usefonttheme{serif}
 %\usecolortheme{wolverine}
% \usecolortheme{rose}
\usefonttheme{structurebold}
}

\setbeamercovered{dynamic}

\title[MathsCast]{Statistics for Computing \\ {\normalsize MA4413 Lecture 2A}}
\author[Kevin O'Brien]{Kevin O'Brien \\ {\scriptsize Kevin.obrien@ul.ie}}
\date{Autumn Semester 2012}
\institute[Maths \& Stats]{Dept. of Mathematics \& Statistics, \\ University \textit{of} Limerick}

\renewcommand{\arraystretch}{1.5}

\begin{document}
Question 1 Part B

Let P signify that a test will give a “positive” result 
Let N signify that a test will give a “negative” result	
Let D signify that the person in question has the disease
Let H signify that the person doesn’t have the disease ( or in other words , is healthy) 

We are asked to determine the following 
1) The probability of a positive test - p(P)
2) The probability that they have the disease given that they have tested positive – p(D|P)
	

We are given the following three pieces of information
 

We know that D and H are complements, so we can work out the probabilities of these too. (P and N are complements also)

 

People who test positive are made up of two groups
	People who test positive and who do have the disease  (P and D)
	People who test positive and who don’t have the disease  ( P and H)

	 

Bayes Rule is given in the Formulae 		 

We can rearrange it as follows 		 

We can now write our equation in terms of all the information we have :

   ANS


For the second part, we simply use Bayes Rule again, using information we have determined previously

  ANS



%==========================================================================%
Spring 2008

a)	The diameter of a screw produced in a factory has a normal distribution with mean 8mm and standard deviation 0.05mm. A screw is defective if its diameter is less than 7.92mm or more than 8.08mm.

i)	Calculate the probability that a screw is defective.
(4 marks)
ii) 	Suppose it is possible to control the variance of the diameter without changing the mean diameter. What is the maximum variance possible such that only 1% of screws are defective?
(4 marks)

%==========================================================================%
Repeat 2007
b) The percentage of carbon in a batch of steel produced using a particular method is normally distributed with mean 2 and variance 0.09. 

i)	Calculate the probability that the percentage of carbon in a batch is less than 2.15

ii)	The batch is high grade if the percentage of carbon is between 1.7 and 2.3. Calculate the probability that the batch is high grade. 

iii)	Suppose that it is possible to control the variance. What is the maximum variance possible such that at least 95% of the batches are high grade?
(10 marks)

%==========================================================================%
Spring 2007 ( Question 3)
b) The percentage of carbon in a batch of steel produced using a particular method is normally distributed with mean 3 and variance 0.16. 

iv)	Calculate the probability that the percentage of carbon in a batch is less than 3.3

v)	The batch is high grade if the percentage of carbon is between 2.5 and 3.5. Calculate the probability that the batch is high grade. 



vi)	Suppose that it is possible to control the variance. What is the maximum variance possible such that at least 95% of the batches are high grade?
(10 marks)

%==========================================================================%
Repeat 2006
	Q3.    (a) 	Suppose an oil exploration company purchases drill bits that have a life span that is approximately normally distributed, with a mean equal to 80 hours and a standard deviation equal to 10 hours.

(i)	What is the probability that a drill bit will fail before 60 hours of use?

(ii)	What is the probability that a drill bit will last between 70 hours and 90 hours?

(iii)	The life span of 95% of drill bits is below what value?


%==========================================================================%
Spring 2006
Q3.    (a)	The breaking strength of a certain type of plastic block is normally distributed with a
mean of 1500kg and standard deviation of 50kg. 

(iv)	What is the probability that a block with have a breaking strength greater than 1570kg?

(v)	What is the probability that a block with have a breaking strength measuring between 1482kg and 1518kg?

(iii)	Determine the maximum load such that no more than 5% of the blocks break?

%==========================================================================%
Repeat 2005 (question 2)
(b)	The compressive strength of concrete for fresh-water exhibition tanks has mean 5000 psi and standard deviation 240 psi. Assuming that the compressive strength is normally distributed, calculate the probability that the compressive strength of a sample of concrete is less than 4900 psi.
 

%==========================================================================%        
Spring 2005 
Q3.    (a)	An important manufacturing process produces cylindrical component parts for the automotive industry. The diameter of these parts is normally distributed with a mean of 5 millimeters and a standard deviation of 0.1 millimeters.

(vi)	What is the probability that a part will have a diameter greater than 5.24mm?
(vii)	What is the probability a part will have diameter measuring between 4.78mm and 4.85mm?
(iii)	The diameter of 99% of the parts is below what value?


%==========================================================================%
MA4104 Business Statistics SPRING 2008

Q1. (a) A tyre manufacturer claims that under normal driving conditions, the tread life of a certain tyre follows a normal distribution with mean 50,000 miles and standard deviation 5000 miles. 

(i) If your tyres wear out at 45,000 miles, would you consider this unusual? Support your answer with an appropriate probability calculation using the normal curve. [ 10 marks ] 

(ii) If the manufacturer sells 100,000 of these tyres and warrants them to last at least 40,000 miles, about how many tyres will wear out before the warranty expires? [ 10 marks ]


%=====================================================%


Question 1(a)

Upper Limit ;  U = \mu + 3 \sigma 
Lower Limit ;  L = \mu - 3 \sigma

Standardisation
Apply the standardisation formula	Z=\frac{x-\mu}{\sigma} 	to both limits

\[ Z_U = \frac{U-\mu}{\sigma} =  \frac{(\mu + 3 \sigma)-\mu}{\sigma} = 3\]
 
Similary

$Z_l=-3$ 

\noindent \textbf{Probability of point being above Upper Limit}

From Murdoch Barnes Tables (page 13)  $P(Z \geq 3)=0.00135$

Probability of point being below Lower Limit


To find   we use the “Property of Symmetry”

“Property of Symmetry” -   for any value A
							
Therefore 

Conclusion: 
Probability of point being outside the 3 Sigma limits is

 + =0.00270 	(i.e. 0.27%)
















Question 1(b)

Upper Limit ;  
LowerLimit ;  

Standardisation
Apply the standardisation formula	 	to both limits

 
Similary

 

Probability of point being above Upper Limit

From Murdoch Barnes Tables (page 13)  

Probability of point being below Lower Limit


To find   we use the “Property of Symmetry”

“Property of Symmetry” -   for any value A
							
Therefore 

Conclusion: 
Probability of point being outside the 3 Sigma limits is

 + =0.04550 	(i.e. 4.55%) Question 1(b)
















Question 2C

Upper Limit ; 80.64		Mean		 	
Lower Limit ; 75.36		Standard Deviation	 

Standardisation
Apply the standardisation formula	 	to both limits

 
Similary

 

Probability of being above Upper Limit

From Murdoch Barnes Tables (page 13)  

Probability of being below Lower Limit


To find   we use the “Property of Symmetry”

“Property of Symmetry” -   for any value A
							
Therefore 

Conclusion: 
Probability of point being outside the specification limits 

 + is equal to

 + =0.2584 	(i.e. 26%)














Question 3A

Upper Limit ; 90		Mean		 	
Lower Limit ; 50		Standard Deviation	 

Standardisation
Apply the standardisation formula	 	to both limits

 
Similary

 

Probability of being above Upper Limit

From Murdoch Barnes Tables (page 13)  

Probability of being below Lower Limit


To find   we use the “Property of Symmetry”

“Property of Symmetry” -   for any value A
							
Therefore 

Conclusion: 
Probability of point being outside the specification limits is

 + is equal to


 + =0.01478  	(i.e. 1.5%)















Question 3B

Upper Limit ; 90		Mean		 	
Lower Limit ; 50		Standard Deviation	 

Standardisation
Apply the standardisation formula	 	to both limits

 
Similary

 

Probability of being above Upper Limit

From Murdoch Barnes Tables (page 13)  

Probability of being below Lower Limit


To find   we use the “Property of Symmetry”

“Property of Symmetry” -   for any value A
							
Therefore 

Conclusion: 
Probability of point being outside the specification limits is

 + is equal to

 + =0.01099  	(i.e. 1.1%)









SPRING 2008

Q1. (a) A tyre manufacturer claims that under normal driving conditions, the tread life of a certain tyre follows a normal distribution with mean 50,000 miles and standard deviation 5000 miles. 

(i) If your tyres wear out at 45,000 miles, would you consider this unusual? Support your answer with an appropriate probability calculation using the normal curve. [ 10 marks ] 

(ii) If the manufacturer sells 100,000 of these tyres and warrants them to last at least 40,000 miles, about how many tyres will wear out before the warranty expires? [ 10 marks ]

Part (i) Solution

Test Value ; 45,000km			Mean		 km	
						Standard Deviation	 km

Find  

Standardisation
Apply the standardisation formula	 	to test value
 

i.e.  = 

To find   we use the “Property of Symmetry”

“Property of Symmetry” -   for any value A

From Murdoch Barnes Tables (page 13)  

Therefore   = 0.1587 

“Complement Rule”		 =1-  for any given value A
	
 =1-  = 0.8413
Conclusion

15.87% of Tyres are expected to last less than 45,000km
	
84.13% of Tyres are expected to last longer than 45,000km




Part (i) Solution

Lower Limit ; 40,000km			Mean		 km	
						Standard Deviation	 km

Find  

Standardisation
Apply the standardisation formula	 	to limit
 

i.e.  = 

To find   we use the “Property of Symmetry”

“Property of Symmetry” -   for any value A
	 

From Murdoch Barnes Tables (page 13)  

Therefore   = 0.02275 

Conclusion
2.275% of Tyres are expected to last less than 40,000km

Of a Batch of 100,000 tyres,  2270 tyres will wear out before the warranty expires.
	





\end{document}
