
%--------------------------------------------------------%
\section{ Combinations and Permutations }

%--------------------------------------------------------%
\frame{
\frametitle{Factorials Numbers}

A factorial is a positive whole number, based on a number $n$ , and which is written as $``n!"$. The factorial $n!$ is defined as follows:

\[n!  =n \times (n-1) \times (n-2) \times \ldots \times 2 \times 1 \]

Remark $n!  =n \times (n-1)!$\\ \bigskip

\textbf{ Example: }

\begin{itemize}
\item $3!  = 3 \times 2  \times 1 = 6 $

\item $4!  = 4 \times 3! = 4 \times 3 \times 2 \times 1 = 24$
\end{itemize}
Remark $0! = 1$ not $0$.


}

%--------------------------------------------------------%
\frame{
\frametitle{Permutations and Combinations}


Often we are concerned with computing the number of ways of selecting and arranging groups of items. \begin{itemize} \item  A \textbf{\emph{combination}} describes the selection of items from a larger group of items.  \item A \textbf{\emph{permutation}} is a combination that is arranged in a particular way.
\end{itemize}

\bigskip
\begin{itemize}
\item Suppose we have items A,B,C and D to choose two items from.
\item AB is one possible selection, BD is another. AB and BD are both combinations.
\item More importantly, AB is one combination, for which there are two distinct permutations: AB and BA.
\end{itemize}
}

%--------------------------------------------------------%
\frame{
\frametitle{Combinations}

\textbf{Combinations: }
The number of ways of selecting $k$ objects from $n$ unique objects is:

\[ ^n C_k = {n!  \over k! \times (n-k)!} \]

In some texts, the notation for finding the number of possible combination is written

\[ ^n C_k =  {n \choose k} \]

}

%--------------------------------------------------------%
\frame{
\frametitle{Example of Combinations}
How many ways are there of selecting two items from possible 5?

\[ ^5 C_2   \left( \mbox{ also }  {5 \choose 2}  \right) =  {5!  \over 2! \times 3!} =  {5 \times 4 \times 3!  \over 2 \times 1 \times 3!} = 10  \]

\bigskip
Discuss how combinations can be used to compute the number of rugby matches for each group in the Rugby World Cup.

}
%--------------------------------------------------------%
\frame{
\frametitle{The Permutation Formula}
The number of different permutations of r items from n unique items is written as $^n P_k$


\[ ^n P_k = \frac{n!}{(n-k)!}\]
}

%--------------------------------------------------------%
\frame{
\frametitle{Permutations}
\textbf{Example:}
How many ways are there of arranging 3 different jobs, between 5 workers, where each worker can only do one job?


\[ ^5 P_3 = \frac{5!}{(5-3)!}  = {5! \over 2!} = 60\]

}



%--------------------------------------------------------%
\frame{
\frametitle{Example of Combinations}

A committee of 4 must be chosen from 3 females and 4 males.

\begin{itemize}
\item In how many ways can the committee be chosen.
\item In how many cans 2 males and 2 females be chosen.
\item Compute the probability of a committee of 2 males and 2 females are chosen.
\item Compute the probability of at least two females.
\end{itemize}
}

%--------------------------------------------------------%
\frame{
\frametitle{Example of Combinations}

\textbf{Part 1}

We need to choose 4 people from 7:

This can be done in

\[
^7 C_4  = {7!  \over 4! \times 3!} =  {7 \times 6 \times 5 \times 4!  \over 4! \times 3!} = 35 \mbox{ ways.}
\]


\textbf{Part 2}

With 4 men to choose from, 2 men can be selected in \[
^4 C_2  = {4!  \over 2! \times 2!} =  {4 \times 3 \times 2!  \over 2! \times 2!} = 6\mbox{ ways.}
\]

Similarly 2 women can be selected from 3 in
\[
^3 C_2  = {3!  \over 2! \times1!} =  {3 \times 2!  \over 2! \times 1!} = 3\mbox{ ways.}
\]

}

%--------------------------------------------------------%
\begin{frame}[fragile]
\frametitle{Using \texttt{R}}
When implementing combination calculations in \texttt{R}, we use the \texttt{choose()} function.

\begin{verbatim}
> choose(5,0)
[1] 1
> choose(5,1)
[1] 5
> choose(5,2)
[1] 10
> choose(5,3)
[1] 10
> choose(5,4)
[1] 5
> choose(5,5)
[1] 1
\end{verbatim}

\end{frame}
%--------------------------------------------------------%
\frame{
\frametitle{Example of Combinations}

\textbf{Part 2}

Thus a committee of 2 men and 2 women can be selected in $ 6 \times 3  = 18 $ ways.\\
\bigskip
\textbf{Part 3}

The probability of two men and two women on a committee is
\[ {\mbox{Number of ways of selecting 2 men and 2 women} \over \mbox{Number of ways of selecting 4 from 7}} = {18 \over 35 }\]

}
%--------------------------------------------------------%
\frame{
\frametitle{Example of Combinations}

\textbf{Part 4}
\begin{itemize}
\item The probability of at least two females is the probability of 2 females or 3 females being selected.
\item We can use the addition rule, noting that these are two mutually exclusive events.
\item From before we know that probability of 2 females being selected is 18/35.
\end{itemize}

}
%--------------------------------------------------------%
\frame{
\frametitle{Example of Combinations}

\textbf{Part 4}
\begin{itemize}
\item We have to compute the number of ways of selecting 1 male from 4 (4 ways) and the number of ways of selecting three females from 2 ( only 1 way)
\item The probability of selecting three females is therefore ${4 \times 1 \over 35} = 4/35$
\item So using the addition rule
\[ Pr(\mbox{ at least 2 females }) = Pr(\mbox{ 2 females }) + Pr(\mbox{ 3 females }) \]
\[ Pr(\mbox{ at least 2 females })  = 18/35 + 4/35 = 22/35 \]
\end{itemize}

}



%--------------------------------------------------------%
\section{ Combinations and Permutations }

%--------------------------------------------------------%
\frame{
	\frametitle{Factorials Numbers}
	
	A factorial is a positive whole number, based on a number $n$ , and which is written as $``n!"$. The factorial $n!$ is defined as follows:
	
	\[n!  =n \times (n-1) \times (n-2) \times \ldots \times 2 \times 1 \]
	
	Remark $n!  =n \times (n-1)!$\\ \bigskip
	
	\textbf{ Example: }
	
	\begin{itemize}
		\item $3!  = 3 \times 2  \times 1 = 6 $
		
		\item $4!  = 4 \times 3! = 4 \times 3 \times 2 \times 1 = 24$
	\end{itemize}
	Remark $0! = 1$ not $0$.
	
	
}

%--------------------------------------------------------%
\frame{
	\frametitle{Permutations and Combinations}
	
	
	Often we are concerned with computing the number of ways of selecting and arranging groups of items. \begin{itemize} \item  A \textbf{\emph{combination}} describes the selection of items from a larger group of items.  \item A \textbf{\emph{permutation}} is a combination that is arranged in a particular way.
	\end{itemize}
	
	\bigskip
	\begin{itemize}
		\item Suppose we have items A,B,C and D to choose two items from.
		\item AB is one possible selection, BD is another. AB and BD are both combinations.
		\item More importantly, AB is one combination, for which there are two distinct permutations: AB and BA.
	\end{itemize}
}

%--------------------------------------------------------%
\frame{
	\frametitle{Combinations}
	
	\textbf{Combinations: }
	The number of ways of selecting $k$ objects from $n$ unique objects is:
	
	\[ ^n C_k = {n!  \over k! \times (n-k)!} \]
	
	In some texts, the notation for finding the number of possible combination is written
	
	\[ ^n C_k =  {n \choose k} \]
	
}

%--------------------------------------------------------%
\frame{
	\frametitle{Example of Combinations}
	How many ways are there of selecting two items from possible 5?
	
	\[ ^5 C_2   \left( \mbox{ also }  {5 \choose 2}  \right) =  {5!  \over 2! \times 3!} =  {5 \times 4 \times 3!  \over 2 \times 1 \times 3!} = 10  \]
	
	\bigskip
	Discuss how combinations can be used to compute the number of rugby matches for each group in the Rugby World Cup.
	
}
%--------------------------------------------------------%
\frame{
	\frametitle{The Permutation Formula}
	The number of different permutations of r items from n unique items is written as $^n P_k$
	
	
	\[ ^n P_k = \frac{n!}{(n-k)!}\]
}

%--------------------------------------------------------%
\frame{
	\frametitle{Permutations}
	\textbf{Example:}
	How many ways are there of arranging 3 different jobs, between 5 workers, where each worker can only do one job?
	
	
	\[ ^5 P_3 = \frac{5!}{(5-3)!}  = {5! \over 2!} = 60\]
	
}



%--------------------------------------------------------%
\frame{
	\frametitle{Example of Combinations}
	
	A committee of 4 must be chosen from 3 females and 4 males.
	
	\begin{itemize}
		\item In how many ways can the committee be chosen.
		\item In how many cans 2 males and 2 females be chosen.
		\item Compute the probability of a committee of 2 males and 2 females are chosen.
		\item Compute the probability of at least two females.
	\end{itemize}
}

%--------------------------------------------------------%
\frame{
	\frametitle{Example of Combinations}
	
	\textbf{Part 1}
	
	We need to choose 4 people from 7:
	
	This can be done in
	
	\[
	^7 C_4  = {7!  \over 4! \times 3!} =  {7 \times 6 \times 5 \times 4!  \over 4! \times 3!} = 35 \mbox{ ways.}
	\]
	
	
	\textbf{Part 2}
	
	With 4 men to choose from, 2 men can be selected in \[
	^4 C_2  = {4!  \over 2! \times 2!} =  {4 \times 3 \times 2!  \over 2! \times 2!} = 6\mbox{ ways.}
	\]
	
	Similarly 2 women can be selected from 3 in
	\[
	^3 C_2  = {3!  \over 2! \times1!} =  {3 \times 2!  \over 2! \times 1!} = 3\mbox{ ways.}
	\]
	
}

%--------------------------------------------------------%
\begin{frame}[fragile]
	\frametitle{Using \texttt{R}}
	When implementing combination calculations in \texttt{R}, we use the \texttt{choose()} function.
	
	\begin{verbatim}
	> choose(5,0)
	[1] 1
	> choose(5,1)
	[1] 5
	> choose(5,2)
	[1] 10
	> choose(5,3)
	[1] 10
	> choose(5,4)
	[1] 5
	> choose(5,5)
	[1] 1
	\end{verbatim}
	
\end{frame}
%--------------------------------------------------------%
\frame{
	\frametitle{Example of Combinations}
	
	\textbf{Part 2}
	
	Thus a committee of 2 men and 2 women can be selected in $ 6 \times 3  = 18 $ ways.\\
	\bigskip
	\textbf{Part 3}
	
	The probability of two men and two women on a committee is
	\[ {\mbox{Number of ways of selecting 2 men and 2 women} \over \mbox{Number of ways of selecting 4 from 7}} = {18 \over 35 }\]
	
}
%--------------------------------------------------------%
\frame{
	\frametitle{Example of Combinations}
	
	\textbf{Part 4}
	\begin{itemize}
		\item The probability of at least two females is the probability of 2 females or 3 females being selected.
		\item We can use the addition rule, noting that these are two mutually exclusive events.
		\item From before we know that probability of 2 females being selected is 18/35.
	\end{itemize}
	
}
%--------------------------------------------------------%
\frame{
	\frametitle{Example of Combinations}
	
	\textbf{Part 4}
	\begin{itemize}
		\item We have to compute the number of ways of selecting 1 male from 4 (4 ways) and the number of ways of selecting three females from 2 ( only 1 way)
		\item The probability of selecting three females is therefore ${4 \times 1 \over 35} = 4/35$
		\item So using the addition rule
		\[ Pr(\mbox{ at least 2 females }) = Pr(\mbox{ 2 females }) + Pr(\mbox{ 3 females }) \]
		\[ Pr(\mbox{ at least 2 females })  = 18/35 + 4/35 = 22/35 \]
	\end{itemize}
	
}
\end{document}