%---------------------------------------------------------------------------%
\frame{
\frametitle{Continuous Random Variables}

\begin{itemize}
\item Probability Density Function
\item Cumulative Density Function
\end{itemize}


If X is a continuous random variable then we can say that the probability of obtaining a \textbf{precise} value $x$ is infinitely small, i.e. close to zero.

\[P(X=x) \approx 0 \]

Consequently, for continuous random variables (only),  $P(X \leq x)$ and $P(X < x)$ can be used interchangeably.

\[P(X \leq x) \approx P(X < x) \]


}

%---------------------------------------------------------------------------%
\frame{
\frametitle{Continuous Uniform Distribution}
A random variable X is called a continuous uniform random variable over the interval $(a,b)$ if it's probability density function is given by

\[ f_{X}(x)  =  { 1 \over b-a}   \hspace{2cm}  \mbox{ when } a \leq x \leq b\]

The corresponding cumulative density function is

\[ F_x(x) = { x-a \over b-a}   \hspace{2cm}  \mbox{ when } a \leq x \leq b\]

}

%-----------------------------------------------------------------------------%

\frame{

The mean of the continuous uniform distribution is

\[ E(X) = {a+b \over 2}\]

\[ V(X) = {(b-a)^2\over12}\]
}

%-----------------------------------------------------------------------------%

\frame{
\frametitle{The Memoryless property}
The most interesting property of the exponential distribution is the \textbf{\emph{memoryless}} property. By this , we mean that if  the lifetime of a component is exponentially distributed, then an item which has been in use for some time is a good as a brand new item with regards to the likelihood of failure.

The exponential distribution is the only distribution that has this property.
}

%--------------------------------------------------------%